\section{Dato: 18.05.2016}
\hrule

\textbf{Omhandler:} Borgerafsnit - særligt interviewundersøgelsen fra Hadsten Sundhedscenter

\textbf{Ansvarlig:} Melissa

\textbf{Dagsorden:}
\begin{itemize}
	\item Vurdering af interviewundersøgelsens gyldighed
	
\end{itemize}

\textbf{Logbog}
\\
\\ \\
Bente opsøges med henblik på hjælp til vurdering af interviewundersøgelsens gyldighed. 
Spørgsmål: kan interviewundersøgelsen godt anvendes, idet den er lidt mangelfuld? F.eks. mangler vi data fra to interviews (spørgsmål 7 og 8 mangler simpelthen).
Møde med Bente: 
Bente mener sagtens, at vi kan inddrage interviewundersøgelsen - det er "bare" en lille undersøgelse, som Hadsten Sundhedscenter har lavet med det formål at høre borgernes og sygeplejerskernes reaktion på videoopkald i pilotprojektet. Så længe vi forholder os til, at undersøgelsen ikke er større, end den er - så kan det sagtens retfærdiggøres, at vi vælger at medtage den.
På baggrund heraf er interviewundersøgelsen medtaget i borgerafsnittet - og der er skrevet et afsnit i metodeafsnittet, hvor vi kritisk reflekterer over dette og diskuterer gyldigheden af interviewundersøgelsen. 