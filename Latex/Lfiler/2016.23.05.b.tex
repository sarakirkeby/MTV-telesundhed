\section{Dato: 23.05.2016}
\hrule

\textbf{Omhandler:} Borgerafsnit, økonomiafsnit, organisationsafsnit, overordnet metodeafsnit og diverse.

\textbf{Ansvarlig:} Melissa, Lise, Jeppe, Jakob, Mohamed, Sara
\textbf{Dagsorden:}
\begin{itemize}
	\item Borgerafsnit færdiggøres og smides i latex
	\item Organisationsafsnit færdiggøres og smides i latex
	\item Teknologi færdiggøres 
	\item Økonomi færdiggøres 
	\item Overordnet metodeafsnit videreudvikles
	\item Projektstyringsafsnit færdiggøres
	\item Diverse mangler, retterlse mm. 
\end{itemize}

\textbf{Logbog}
\\
\\ \\
Sara har arbejdet videre med organisationsafsnittet. Jakob har arbejdet videre med økonomiafsnittet, og har desuden været forbi Bente for at høre, hvor meget der skal inkluderes direkte i afsnittet af beregninger, og hvor meget der hører til i bilag. Bente ønsker ingen udregninger i afsnittet - dette skal lægges som bilag. 
Lise har arbejdet med bilagsafsnittet omkring projektstyring og har desuden hjulpet med at få indsat dokumenter i latex.
Jeppe og Mohamed har arbejdet med teknologiafsnittet. Jeppe har derudover lavet en masse generelle rettelser og har arbejdet med at få indsat dokumenter i latex.
Melissa har arbejdet videre med borgerafsnittet og har desuden lavet interessentanalysen. Melissa var forbi Bentes kontor for at drøfte, hvordan interessentanalysen præcis skule gribes an. Ifølge Bente er det op til os, hvor "grundig" interessentanalysen er - vi skal blot have styr på alle interessenter og kunne redegøre for, hvorfor vores analyse er uarbejdet, som den er. 

Jeppe og Lise har læst bogerafsnittet igennem, og sammen med Melissa gennemgås afsnittet kritisk. Borgerafsnittet rettes til på baggrund heraf. 
Ift. grading af videnskabelige artikler var Jeppe og Melissa forbi Samuels kontor, hvor grading blev diskuteret. Ifølge Samuel skal vi ikke udarbejde en tabel over samtlige anvendte artikler og deri grade dem, men i stedet løbende ved henvisning til konkrete artikler kritisk reflektere over, hvad type studie artiklen er, hvor mange participants, hvor repræsentativ, oprindelsesland mm. Hvis vi er kritisk reflekterende og ganske kort kommenterer på dette, hver gang vi henviser til en artikel, så grader vi. 
Tabeller er bedst til fremvisning af tal.
