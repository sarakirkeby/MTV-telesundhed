\section{Dato: 18.05.2016}
\hrule

\textbf{Omhandler:} Borgerafsnit, grading af artikler og referenceliste

\textbf{Ansvarlig:} Melissa, Lise, Jeppe
\textbf{Dagsorden:}
\begin{itemize}
	\item Gennemlæsning af borgerafsnit
	\item Grading af artikler 
	\item Referenceliste
	
\end{itemize}

\textbf{Logbog}
\\
\\ \\
Jeppe og Lise har læst bogerafsnittet igennem, og sammen med Melissa gennemgås afsnittet kritisk. Borgerafsnittet rettes til på baggrund heraf. 
Ift. grading af videnskabelige artikler var Jeppe og Melissa forbi Samuels kontor, hvor grading blev diskuteret. Ifølge Samuel skal vi ikke udarbejde en tabel over samtlige anvendte artikler og deri grade dem, men i stedet løbende ved henvisning til konkrete artikler kritisk reflektere over, hvad type studie artiklen er, hvor mange participants, hvor repræsentativ, oprindelsesland mm. Hvis vi er kritisk reflekterende og ganske kort kommenterer på dette, hver gang vi henviser til en artikel, så grader vi. 
Tabeller er bedst til fremvisning af tal.
