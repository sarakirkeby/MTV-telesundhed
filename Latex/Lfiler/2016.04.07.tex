\section{Dato: 07.04.2016}
\hrule

\textbf{Omhandler:}

\textbf{Ansvarlig:} Sara, Melissa, Lise, Mohamed, Jeppe og Jakob

\textbf{Dagsorden:}
\begin{itemize}
	\item Opdeling
	\item Møde med Appinux(økonomi - dataopsamling)
	\item Deadlines
	\item Litteratursøgning
	\item Patient/etik (udkast)
	\item To do liste
	\item Problemformulering
	\item Template (se vejledning)
\end{itemize}

\textbf{Logbog}
\\


\textbf{Opdeling}
\\
Vi snakkede om at dele os op i to hold, hvor det er de samme 3 personer der dækker to af områderne og de 3 andre der dækker de to andre områder. På den måde er det lettest at arrangere møder og arbejdsopgaver grupperne imellem.\\
Vi snakkede om at dele gruppen op i teknologi/økonomi og organisation/patient/etik.\\
Pigerne tager sig organisation/etik delen.\\
Drengene tager sig af teknologi/økonomi delen.\\
På den måde får vi hver især 2 hovedområder, og så bliver vi også tildelt 2 underområder hver, således at vi alle har en finger med i hvert delelement af MTV'en. \\
Vi holder et ugentligt møde, hvor vi holder et lille oplæg og forklare, hvor langt vi er. Derefter kommer man med indspark til hinanden og giver gode råd til diverse ting og sager.

Møde kommer til at ligge fast torsdag kl. 10.15\\

Herunder ses projektlederne for de 4 elementer af MTV'en: \\
Teknologi:
Jeppe\\

Økonomi:
Jakob\\

Patient/etik:
Melissa\\

Organisation:
Sara\\


\textbf{Møde med Appinux(økonomi - dataopsamling)}
\\
Vi har tilføjet nogle økonomiske spørgsmål til listen, samt noget omkring dataopsamling. Vi er nu klar til mødet med Appinux. 

\textbf{Deadlines}
\\
04.05.2016 = foreløbig projekt aflevering
27.05.2016 = endelig projekt aflevering

\textbf{Litteratursøgning}
\\
Få fat Rasmus Thorbjørn manden og arranger et møde til litteratursøgning.
Melissa har taget kontakt til ham angående et møde enten den 13. eller 14. april.\\
Når vi har fået lavet de fokuserede spørgsmål, kan vi i hver gruppe lave en mere dybdegående litteratursøgning. 

\textbf{Etik (udkast)}
\\
Vi har sendt det til Preben og får feedback på tirsdag. 

\textbf{To do liste}
\\
Liste over dokumenter man skal have kigget på hurtigst muligt:
\begin{itemize}
	\item Kommunernes strategi for telesundhed
	\item Det nære sundhedsvæsen
	\item Kandidatspeciele for Viborg Kommune
	\item Reference arkitektur
\end{itemize}

\textbf{Problemformulering}
\\
Vi laver problemformuleringen sammen. Hvilke spørgsmål skal besvares i MTV'en?\\ \\

Fokuserede spørgsmål: \\

Overordnede fokuserede spørgsmål: \\
Fungerer Appinux' løsning optimalt i forhold til målet med implementeringen af telesundhed i Favrskov Kommune?

Hvordan er brugernes reaktion og hvad skal man være opmærksom på?

Teknologi: \\
Hvilke forudsætninger er der for at video tele conferencing i telesundhed fungerer optimalt?\\
\begin{itemize}
	\item Bredbånd/infrastruktur
	\item Udstyr
	\item Codec
	\item Sikkerhedskrav
	\item Brugervenlighed
\end{itemize}

Hvilke forudsætninger opfylder Appinux' løsning i forhold til video tele conferencing?
\begin{itemize}
	\item ???
	\item ???
	\item ???
	\item ???
	\item ???
\end{itemize}


Økonomi: \\
Identificere udgifter/omkostninger.\\
Omkostninger for patient/organisation - forskel mellem de to scenarier.
\\

Patient/etik:
Hvilke forudsætninger er der for at video tele conferencing i  telesundhed fungerer optimalt?\\
\begin{itemize}
	\item Brugervenlighed
	\item Undervisning
	\item Villighed
\end{itemize}

Hvad er borgernes reaktion?\\

Organisation:
Hvilke forudsætninger er der for at video tele conferencing i  telesundhed fungerer optimalt?\\
\begin{itemize}
	\item Brugervenlighed
	\item Undervisning - Superbruger
	\item Villighed
	\item Ledelsesopbakning
	\item Effektivitet
	\item Resourcer - arbejdsgang
\end{itemize}

Hvad er de sundhedsprofessionelles reaktion?\\

\textbf{Template (se vejledning)}
\\
Lise laver et udkast til en template.
