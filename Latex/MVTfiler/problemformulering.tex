\chapter{Problemformulering}

Netplan Care og Favrskov Kommune er i gang med et innovationssamarbejde om udviklingen af en kommunal digital velfærdsteknologisk sundhedsstrategi for Telesundhed. 
\\ \\
Telesundhed dækker over digitale velfærdsydelser på mobil- og bredbåndsnettet, hvor sundhedsfaglig dialog og behandling ved brug af den digitale infrastruktur muliggør, at borgere smidigt og omkostningseffektivt kan komme i kontakt med sundhedsvæsenet.    
\\ \\
Video er den mest komplekse løsningskomponent i forhold til telesundhedsløsninger. En af de digitale velfærdsteknologier Favrskov Kommune arbejder med at implementere er Virtuel hjemmepleje, som i høj grad benytter videos som et redskab til kommunikation mellem borger og sundhedsprofessionel. 
\\ \\
Sundhedsteknologistuderende fra Aarhus Ingeniørhøjskole udarbejder i samarbejde med Netplan Care og Favrskov Kommune en Medicinsk Teknologi Vurdering af videobaserede løsninger for Virtuel hjemmepleje. Analysen skal især afdække de teknologiske aspekter samt borgeres reaktioner på video som telesundhedsløsning. Ligeledes vil aspektet om organistionen være i fokus. 
\\
hvilke forudsætninger skal der til for at video fungerer i telesundhedsløsninger? 
hvilket behov kan videi dække 
hvordan er brugernes reaktion og hvad skal man være opmærksom på, opdelt på de sundhedsprogessionelle og borgerne. 