\chapter{Abstract}

\textbf{Background}\\
The background for this mini-HTA is to illustrate the value of \textit{video conferencing} in virtual home care in\textit{Pilotprojekt Videokommunikation} in Sundhedscenter Hadsten as a part of the development of a strategy for tele health in Favrskov Municipality.

\textbf{Materials and methods}\\
The method consists of literature search in academic/scientific? databases, a qualitative interview study and meetings with significant stakeholders.

\textbf{Results}\\
The results point out problems with coverage in parts of Favrskov Municipality, and that statutory safety requirements exist. Results indicate great satisfaction and acceptance among the citizens who have received \textit{video conferencing}. Furthermore, the results point out that the organisational implementation was successful. It was not possible to obtain unambiguous results, but the most significant expenses have been covered.

\textbf{Discussion}\\
The solution from Appinux has been compared to requisites required by infrastructure and safety (bør måske omformuleres?). The technology has been discussed with a view to identify essential focus points. Satisfaction, acceptance and comfort among the citizens from the pilot project has been compared to other studies in this field. The implementation process in Sundhedscenter Hadsten has been discussed and evaluated. An economic statement of resources has been set up, and the economic impact (flertal?) has been discussed on this basis.

\textbf{Conclusion}\\
