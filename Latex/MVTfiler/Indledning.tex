\chapter{Indledning}

\section{Baggrund}
Overordnet om telesundhedsløsninger - postitiv/negativ. Hvad er det? (http://sundhedsdatastyrelsen.dk/da/rammer-og-retningslinjer/om-digitaliseringsstrategi/telemedicin-og-telesundhed) Hvilket former er der osv. 


Hvad gør DK for disse løsninger - handleplan indenfor telesundhed på national plan og i kommunerne. Ny digitaliseringsstrategi 2016-2020
\\ \\
Demografien i dk i forhold til ældre - vi skal have noget statistik på dette område (tal). også tal på hvormange der bruger hjemmeplejen. Hvad er deres udgangspunkt ift. teknologi. 
\\ \\
Telesundhed = Virtuel hjemmepleje. World wide - eksempler hvor det er blevet benyttet og virker. eksempler på kommuner, der har forsøgt sig med virtuel hjemmepleje - samt hvilke løsninger/leverandører. (videnskabelige artikler, er der nogle, der har lavet nogle studier) 
ulemper = infrastruktur.
 

\section{Formål}
Netplan Care og Favrskov Kommune er i gang med et innovationssamarbejde om udviklingen af en kommunal digital velfærdsteknologisk sundhedsstrategi for Telesundhed. 
\\ \\
Telesundhed dækker over digitale velfærdsydelser på mobil- og bredbåndsnettet, hvor sundhedsfaglig dialog og behandling ved brug af den digitale infrastruktur muliggør, at borgere smidigt og omkostningseffektivt kan komme i kontakt med sundhedsvæsenet.    
\\ \\
Video er den mest komplekse løsningskomponent i forhold til telesundhedsløsninger. En af de digitale velfærdsteknologier Favrskov Kommune arbejder med at implementere er Virtuel hjemmepleje, som i høj grad benytter videos som et redskab til kommunikation mellem borger og sundhedsprofessionel. 
\\ \\
Sundhedsteknologistuderende fra Aarhus Ingeniørhøjskole udarbejder i samarbejde med Netplan Care og Favrskov Kommune en Medicinsk Teknologi Vurdering af videobaserede løsninger for Virtuel hjemmepleje. Analysen skal især afdække de teknologiske aspekter samt borgeres reaktioner på video som telesundhedsløsning. Ligeledes vil aspektet om organistionen være i fokus. 

\section{Fokuserede spørgsmål}
\begin{itemize}
	\item Hvilke forudsætninger skal der til for at video fungerer i telesundhedsløsninger? 
	\item Hvilket behov kan video i telesundhedsløsninger dække?
	\item Hvordan er brugernes reaktion og hvad skal man være opmærksom på, opdelt på de sundhedsprofessionelle og borgerne. 
\end{itemize}

