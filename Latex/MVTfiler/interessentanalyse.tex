\documentclass[10pt,a4paper]{article}
\usepackage[utf8]{inputenc}
\usepackage[style=numeric,urldate=long, maxbibnames=6,sortcites=true,sorting=none, backend=bibtex]{biblatex}
\addbibresource{ref2.bib}
\usepackage{graphicx}
\begin{document}
	\section{Interessentliste}
	\textbf{Appinux}\\
	Leverandør af videokonferencersystemet i Favrskov Kommune. Appinux er en social- og sundhedsplatform \cite{appinuxwebsite}.
	Interesse: XX \\ \\
	\textbf{Sundhedsstyrelsen}\\
	Den øverste sundhedsfaglige instans i Danmark.\\ Interesse:Primæropgaver er sundhedsfremme, forebyggelse og sygdomsbehandling \cite{sstyr}.\\ \\
	\textbf{Digitaliseringsstyrelsen}\\
	Ansvarlig for udarbejdelsen af fællesoffentlig digitaliseringstrategi frem mod 2020, hvor datadeling, datasikkerhed og it-infrastruktur er temaer \cite{digst1}.\\
	Interesse: At understøtte de teknologiske muligheder for smartere og mere sikker deling ad data mellem borger og offentlig sektor \cite{digst2}.\\ \\
	\textbf{Byrådet}\\
	Byrådet: Opdragsholder og betalende part. Byrådet i Favrskov Kommunen har vedtaget en række politikker, som kommunens serviceområder styres efter \cite{favrskovkommune}. Ifølge Sundhedspolitikken for Favrskov Kommune 2016-2019 er fokus blandt andet mere sundhed i det nære og et aktivt liv i sundt miljø.\\
	Interesse: At fremme sundhed i kommunen med henblik på, at de borgere, der kan selv, skal selv \cite{favrskovkommune2}.\\ \\
	\textbf{Ældreområdet}\\
	Ældreområdet i Favrskov Kommune har en målsætning om, at ældre borgere i kommunen kan få hjælp ved behov \cite{favrskovkommune3}.\\
	Interesse: At sikre høj serviceniveau for borgere i hjemmeplejen i Favrskov Kommune. \\ \\
	\textbf{Sundhedscenter Hadsten}\\
	Sundhedscentret i Hadsten samler en række tilbud til kommunens borgere. Sundhedscenteret huser de sundhedsprofessionelle, der arbejder med borgere i hjemme- og sygeplejen. De sundhedsprofessionelle er direkte bruger af videokonferencesystemet.\\
	Interesse: Gevinstrealisering i form af sparet kørsel og mindre tidsforbrug. Tilbud til borgeren om hjemme- og sygepleje uden unødige forstyrrelser i borgerens hverdag og hjem. Opretholdelse af borgertilfredshed og kvalitetsniveau ved hjemme- og sygepleje (Kilde: bilag - mail fra Karin Juhl).\\ \\
	\textbf{Borger}\\
	Borgere er de personer, som har brug for hjemmepleje. Borgerne er direkte brugere af videokonferencesystemet.\\
	Interesse: Modtage hjemmepleje med høj serviceniveau.\\ \\
	\textbf{Netplan}\\
	Konsulentfirma\cite{netplan}, hvis opgave er at vejlede og rådgive Favrskov Kommune gennem tilblivelsen af en fælles strategi for telesundhed i Favrskov Kommune.\\
	Interesse: At medvirke til udviklingen af en fælles kommunal strategi for telesundhed i Favrskov Kommune (kilde: bilag - mødereferat).\\ \\
	\textbf{TDC}\\
	Leverandør af internetopkobling på borgerens tablet. Virtuel hjemmepleje med videokonference er afhængig af en stabil og hurtig internetopkobling, hvilket stiller høje krav til leverandøren. \\
	Interesse: At udbyde den bedste dækning.
	\printbibliography[title=Referencer]
\end{document}