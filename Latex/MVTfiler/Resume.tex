\chapter{Resume}

\textbf{Baggrund}\\
Baggrunden for denne mini-MTV er at belyse betydningen af videoopkald i virtuel hjemmepleje i \textit{Pilotprojekt Videokommunikation} i Sundhedscenter Hadsten som et led i udviklingen af en strategi for telesundhed i Favrskov Kommune.

\textbf{Materiale og metoder}\\
Metoden har bestået af litteratursøgning i videnskabelige databaser, en kvalitativ interviewundersøgelse og møder med væsentlige interessenter.
 
\textbf{Resultater}\\
Resultater peger på dækningsproblemer nogle steder i Favrskov Kommune, og at der findes lovmæssige sikkerhedskrav. Resultater tyder på en stor tilfredshed og accept blandt borgere, der har modtaget videoopkald. Desuden peger resultater på, at den organisatoriske implementering har været succesfuld. Det har ikke været muligt at nå frem til entydige økonomiske resultater, men de væsentligste omkostninger er belyst. 

\textbf{Diskussion}\\
Appinux’ løsning er holdt op i mod forudsætninger for infrastruktur og sikkerhed. Teknologien er desuden diskuteret med henblik på at identificere særlige fokuspunkter. Tilfredshed, borgeraccept og tryghed blandt borgere fra \textit{Pilotprojekt Videokommunikation} er sammenlignet med øvrige studier på området. Implementeringsprocessen i Sundhedscenter Hadsten er diskuteret og vurderet. En økonomisk ressourceopgørelse er opstillet, og de økonomiske betydninger er diskuteret på baggrund heraf.

\textbf{Konklusion}\\
Det kan konkluderes, at Appinux’ løsning med videoopkald i Favrskov Kommune tildels imødekommer de teknologiske forudsætninger. De borgermæssige rammer for tilfredshed, borgeraccept og tryghed er opfyldt og organisatorisk har modtagelsen af ændringer i arbejdsgange være overvejende positiv. Økonomisk set er der ikke et entydigt svar på, hvorvidt implementering af videoopkald i Favrskov Kommune er rentabelt.