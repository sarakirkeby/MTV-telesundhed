\chapter{Etik}
Ingeniøren i sundhedsteknologi er underlagt en personlig og faglig integritet, hvorunder professionsetik er en væsentlig faktor i opfyldelsen af de faglige idealer forbundet med udviklingen og implementeringen af en given sundhedsteknologi.

I brugen af virtuel hjemmepleje gør centrale professionsetiske principper sig gældende. Dette kapitel berører de etiske refleksioner, der relaterer sig til brugen af virtuel hjemmepleje. Kapitlets etiske principper har afsæt i Det etiske hjul af Jørgen Husted \cite{etiskhjul}. 

Det absolutte princip for sundhedsprofessionelle udspringer af pligtetikken og befordrer, at sundhedsprofessionelle altid skal respektere autonomi for brugeren \cite{mtv}. Brugen af virtuel hjemmepleje er et tilbud om levering af virtuelle ydelser, som borgeren er i sin fulde ret til at fravælge. I et sådan konkret tilfælde må plejepersonalet nødvendigvis tilgå borgeren fra et dydsetisk perspektiv, hvor borgeren gennem undervisning og læring opnår selvbestemmelse gennem pædagogiske og kommunikative strategier. Disse strategier skal ligeledes bringes i anvendelse for at lindre lidelse, fremmedgørelse og ubehag hos borgeren. En borger skal til enhver tid være velinformeret omkring samtlige borgernære aspekter vedrørende brugen af virtuel hjemmepleje. 

At implementere virtuel hjemmepleje medfører konsekvenser for to brugerflader; borgeren og den sundhedsprofessionelle. Dydsetiske principper er dermed ikke begrænset til borgeren, men er en arbejdsopgave som også de sundhedsprofessionelle er nødt til at påtage sig, så det bliver udviklet i forbindelse med arbejdet med virtuel hjemmepleje. De sundhedsprofessionelle besidder individuelle muligheder for personlig udvikling, og disse muligheder må og skal udvikles i arbejdet med virtuel hjemmepleje, så kvaliteten i det sundhedsprofessionelle arbejde ikke forsvinder. At kende til teknologien og dens mange muligheder er en forudsætning for tilfredsstillende implementering. Dette kendskab opnås gennem læring og træning af medarbejderne \cite{mtv}. 

Retfærdighedsprincippet forsøges ligeledes opnået gennem implementeringen og brugen af virtuel hjemmepleje, idet et centralt mål er frigivelse af ressourcer i form af tid og varme hænder, så en retfærdig fordeling af ressourcer efter behov kan foregå. Her er det afgørende, at samtlige relevante omkostninger forbundet med virtuel hjemmepleje er afdækket, således at denne ikke pludselig kræver flere ressourcer og dermed bryder med retfærdighedsprincippet. Den retfærdige fordeling af ressourcer skal altid harmonere med idealet om at opretholde kvaliteten i danske sundhedsvæsen \cite{mtv}.

Ikke at skade borgeren er en professionsetisk overvejelse forankret i pligtetikken \cite{mtv}. Her er det især essentielt at nævne sikkerhed i forhold til behandling af personfølsomme oplysninger ved brugen af virtuel hjemmepleje. Brugen af virtuel hjemmepleje skal altid foregå med sikkerhedskrav, der som minimum forhindrer direkte skade af borgeren i form af datasikkerhed.

De væsentligste etiske principper, der kan fremdrages i udbredelsen af virtuel hjemmepleje, er:
\begin{itemize}
	\item Autonomi for borgeren
	\item Fremme autonomi for borgeren gennem kommunikation, undervisning og læring
	\item Fremme personlig udvikling af sundhedsprofessionelle gennem kommunikation, undervisning og læring
	\item Fremme en retfærdig fordeling af ressourcer 
	\item Undgå skade af borgeren i form af insufficient datasikkerhed
\end{itemize}





