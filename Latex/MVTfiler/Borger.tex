\chapter{Borger}

\section{Indledning}
I dette afsnit fokuseres på borgeraspektet i forhold til indførelse af virtuel hjemmepleje i Favrskov Kommune, og der tages især udgangspunkt i ’Pilotprojekt Videokommunikation’ fra Sundhedscenter Hadsten. 

Der gives indledningsvist en introduktion til målgruppen for levering af virtuel hjemmepleje. En klar borgerkarakteristik er nødvendig, idet borgeraspektet afhænger heraf. Definitionen tager udgangspunkt i ’Pilotprojekt Videokommunikation’ fra Sundhedscenter Hadsten, men er ikke afgrænset hertil. 

Formålet med afsnittet er at belyse borgernes oplevelser og erfaringer med brugen af virtuel hjemmepleje i pilotprojektet i Favrskov Kommune. Dette belyses ud fra en strutkureret interviewundersøgelse fra pilotprojektet sammenholdt med videnskabelige studier fra andre lande samt øvrig materiale og viden indhentet gennem møder med interessenter.

Hernæst fremlægges de væsentligste resultater og disse inddrages i en analyse af og diskussion. Afsnittet afsluttes med en konklusion på resultaterne og dermed en besvarelse af det fokuserede spørgsmål.



\subsection{Fokuserede spørgsmål}
\begin{itemize}
	\item Hvilke borgermæssige betydninger er der ved implementering og drift af virtuel hjemmepleje med videokonference i Favrskov Kommune? Spørgsmålet søges besvaret med udgangspunkt i følgende punkter:
	\begin{itemize}
	\item Tilfredshed
	\item Borgeraccept
	\item Tryghed
\end{itemize}
\end{itemize}

\section{Målgruppe}
Målgruppen er borgere i ældregruppen visiteret til hjemmehjælp karakteriseret ved, at hjemmehjælpen i realiteten ikke kræver fysisk tilstedeværelse af en medarbejder. Ergo er målgruppen ældre borgere, der modtager hjælp til at udføre opgaver, som disse med rette påmindelse og støtte selv kan udføre. En klar og entydig, aldersmæssig afgrænsning af begrebet ”ældre” synes svær at finde. Denne mini-MTV læner sig op ad Kommunernes Landsforening og afgrænser dermed ”ældre” til at omfatte borgere på 65 år eller derover\cite{KL}. 

Visitationen af virtuel hjemmehjælp med henblik på følgende ydelser: Medicinadministration og Mellemmåltider. Af borgere med tilbud om ovenstående ydelser er kun inkluderet de, der er i stand til at betjene en tablet \textbf{(kilde: bilag: evalueringsmøde)}.

\section{Metode}
Data og informationer anvendt i borgerafsnittet er indhentet ved litteraturstudie i videnskabelige databaser, generel dataindsamling samt empirisk dataindsamling i form af en interviewundersøgelse fra ’Pilotprojekt Videokommunikation’ fra Sundhedscenter Hadsten. For en dybdegående beskrivelse af metoden henvises til afsnittet Metode.

Specifikke emneord: Home Telemedicine, Telemedicine, Tele Care, Health Care, Tele Health Care.

\section{Resultater}
I dette afsnit fremlægges de resultater i forbindelse med virtuel hjemmepleje, som relaterer sig til følgende forhold:
\begin{itemize}
	\item Tilfredshed
	\item Borgeraccept 
	\item Tryghed
\end{itemize}

\subsection{Tilfredshed}
Resultater tyder på en høj tilfredshed blandt borgere, der har modtaget virtuel hjemmepleje i form af videoopkald. Ifølge et norsk systematisk review “Virtual Visits in Home Health Care for Older Adults” fra 2014 var tilfredsheden med kvaliteten i hjemmeplejen højere blandt borgere, der modtog virtuel hjemmepleje sammenlignet med borgere, der modtog traditionel fysisk hjemmepleje\cite{Baf2}. Ligeledes viste et pilotstudie i Australien fra 2009 blandt ni borgere høj grad af tilfredshed med levering af virtuel hjemmepleje i en periode på seks måneder. Fem ud af otte adspurgte borgere var meget tilfredse og de resterende tre borgere noget tilfredse med videoopkaldene. Ingen af de deltagende borgere var neutrale eller utilfredse med videoopkaldene. Formålet med pilotprojektet var at vurdere praktisk funktionalitet, egnethed, sikkerhed samt omkostningerne ved levering af ydelsen medicinadministration via videoopkald\cite{wade}. 

Ifølge et amerikansk studie, hvor borgeres tilfredshed på baggrund af oplevede fordele og ulemper ved virtuel hjemmepleje blev undersøgt, var tilfredsheden høj, især i forhold til muligheden for vejledning og instruktion ved medicintagning. Studiet blev udført ved spørgeskemaer og efterfølgende individuelle interviews via telefon \textbf{(kilde: Home telehealth: Patient satisfaction, program functions, and the challanges for the care coordinator, nr. 88 i kandidatspeciale).}.

I forlængelse heraf indikerede interviewundersøgelsen fra ’Pilotprojekt Videokommunikation’ fra Sundhedscenter Hadsten samme tendens, idet tre ud af fire adspurgte borgere angav høj tilfredshed med videoopkaldene. Borgerne var samlet set positive over videoopkaldene, og fandt det nye virtuelle tiltag spændende \textbf{(kilde: interviewundersøgelse)}.

I modsætning hertil blev der i det engelske systematiske review ”Telemedicine versus face to face patient care: Effects on professional practice and health care outcomes” fra 2000 ikke fundet signifikant forskel på tilfredsheden blandt modtagere af virtuelle besøg sammenlignet med modtagere af fysiske hjemmeplejebesøg\cite{Paf2}. Dette var ligeledes gældende i et Hollandsk studie fra 2007-2008, hvor formålet var at undersøge borgeres tilfredshed med virtuelle besøg. Studiets resultater viste ingen forskel i tilfredsheden i de virtuelle besøg sammenlignet med konventionelle hjemmeplejebesøg \textbf{[Kilde: Van Offenbeek og Boonstra (nr. 49 i kandidatspeciale)]}. Samme resultat fremkom i 2015 fra et mixed method studie ”Evaluering og dokumentation af telesundhed i kommunal hjemmepleje/sygepleje” fra Viborg Kommune om borgeres tilfredshed samt oplevelser med virtuel hjemmepleje ved medicinadministration sammenlignet med konventionel fysisk hjemmeplejebesøg. I dette studie blev der ikke fundet signifikant forskel i den samlede tilfredshedsscore blandt borgere, der modtog virtuel hjemmepleje og borgere, der modtog konventionel fysisk hjemmepleje\cite{kandidat}.

\subsubsection{Borgeraccept}
Borgeraccept retter fokus mod, hvorvidt borgerne accepterede anvendelsen af videoopkald som alternativ til konventionel fysisk hjemmepleje. Et belgisk systematisk review “Telenursing for the elderly. The case for care via video-telephony” fra 2001 havde til formal at diskutere mulighederne for anvendelsen og levering af virtuel hjemmepleje via videotelefoni. Her blev det påpeget, at videotelefoni blev taget godt imod på baggrund af den visuelle kontakt\cite{telenursing}. I forlængelse heraf viste resultater fra dette systematiske review ligeledes, at borgeraccepten voksede proportionalt med erfaring med videoopkaldene. Jo bedre erfaring med teknologien blandt borgerne, desto større accept af virtuel hjemmepleje.

I interviewundersøgelsen fra ’Pilotprojekt Videokommunikation’ fra Sundhedscenter Hadsten angav tre ud af fire borgere, at de oplevede frihed i forbindelse med den virtuelle hjemmepleje \textbf{(kilde: interviewundersøgelsen). 
} 
Oplevelsen af frihed ved virtuel hjemmepleje blev ligeledes undersøgt i ”Evaluering og dokumentation af telesundhed i kommunal hjemmepleje/sygepleje” fra Viborg Kommune. Her gav flertallet af borgere udtryk for frihed, idet fleksibiliteten af tidspunktet for levering af virtuelle hjemmeplejebesøg var høj. Desuden gav borgere udtryk for, at virtuelle hjemmeplejebesøg blev leveret mere regelmæssigt end fysiske besøg\cite{kandidat}.
 
I kontrast hertil påpegede andre borgere i ”Evaluering og dokumentation af telesundhed i kommunal hjemmepleje/sygepleje” fra Viborg Kommune at være bundet af de virtuelle hjemmeplejebesøg, idet levering af virtuelle hjemmeplejebesøg forudsatte, at borgeren skulle være i eget hjem og klar ved skærmen på et bestemt klokkeslæt. Desuden udtrykte borgere fra Viborg Kommune utilfredshed, hvis ikke videoopkaldet var planlagt på et fast tidspunkt\cite{kandidat}.
 
I et australsk pilotprojekt ”Videophone delivery of medication management in community nursing” var det muligt for en borger at modtage videoopkald før arbejdets start, hvorved følelsen af frihed og fleksibilitet ved virtuelle hjemmeplejebesøg var større end ved fysisk hjemmeplejebesøg\cite{wade}.
 
Resultater fra et pilotprojekt i Viborg Kommune gennemført i 2013 med afprøvning af videoopkald som alternativ til traditionel fysisk hjemmeplejebesøg viste, at borgeren oplevede en mindre grad af stigmatisering, idet virtuel hjemmepleje muliggjorde diskretion for borgeren. Borgeren kunne i fuld fortrolighed modtage konkrete ydelser, uden at hjemmeplejens bil var parkeret uden for borgerens hus\cite{kandidat}.

\subsection{Tryghed}
\subsubsection{Individuelle forhold}
Individuelle oplevelser i forbindelse med virtuel hjemmepleje peger overordnet på en stor tilfredshed med videoopkald blandt borgere. I interviewundersøgelsen fra ’Pilotprojekt Videokommunikation’ fra Sundhedscenter Hadsten angav tre ud af fire borgere, at virtuel hjemmepleje gav en følelse af tryghed, idet virtuel hjemmepleje i modsætning til et telefonopkald gav mulighed for en visuel kontakt mellem borgeren og den sundhedsprofessionelle. En borger udtrykte endvidere, at det var rart at kunne sætte ansigt på den pågældende sundhedsprofessionelle \textbf{(kilde: interviewundersøgelsen)}.

Ifølge et norsk systematisk review ”Virtual Visits in Home Health Care for Older Adults” fra 2014 oplevede borgerne en formindskelse i ensomhed, en forbedret psykosocial kontakt, en formindskelse i følelsen af at være isoleret, en følelse af tryghed og sikkerhed og virtuelle besøg skabte desuden en følelse af være ”cared for”\cite{Baf2}.  

\subsubsection{Kommunikative forhold}
Ifølge det systematiske review “Virtual Visits in Home Health Care for Older Adults” oplevede borgerne en koncentreret kommunikation med sygeplejerskerne. Følelsen af personlig kontakt var højere blandt borgere, der modtog virtuelle besøg sammenlignet med borgere, der modtog fysiske hjemmeplejebesøg\cite{Baf2}. 

I interviewundersøgelsen fra ’Pilotprojekt Videokommunikation’ fra Sundhedscenter Hadsten fortalte en borger, at kommunikationen med en sygeplejerske via videoopkald var positiv, og borgeren oplevede at få det bedre efter samtalen via videoopkald med sygeplejersken \textbf{(kilde: interviewundersøgelsen)}. 

Resultater fra ”Evaluering og dokumentation af telesundhed i kommunal hjemmepleje/sygepleje” i Viborg Kommune viser blandede præferencer ved levering af virtuelle hjemmeplejebesøg sammenlignet med fysiske hjemmeplejebesøg. Ifølge individuelle interviews med borgere fremkom det, at nogle borgere oplevede relationen med den sundhedsprofessionelle som mere menneskelig og naturlig ved fysiske hjemmeplejebesøg. I modsætning hertil angav andre borgere i de individuelle interviews at foretrække virtuelle hjemmeplejebesøg\cite{kandidat}.

\section{Diskussion}
Med udgangspunkt i ovenstående resultatafsnit tyder det generelt på en høj tilfredshed blandt borgere, der har modtaget virtuel hjemmepleje. I de respektive studier fremkom det, at borgere og patienter oplevede virtuel hjemmepleje som et positivt alternativ til konventionel fysisk hjemmepleje\cite{wade},\cite{Baf2},\cite{kandidat}, \textbf{(kilde: Home telehealth: Patient satisfaction, program functions, and the challanges for the care coordinator, nr. 88 i kandidatspeciale)}, \textbf{Interviewundersøgelse}. Dog er det væsentligt at understrege, at flere af studierne konkluderede mangel på evidens. Det systematiske review Effectiveness of telemedicine: A systematic review of reviews konkluderede således, at store, stringente undersøgelser med fokus på patientperspektiv er en nødvendighed for underbygge effekten af telemedicinske interventioner\cite{Ekeland}. \textbf{Flere kilder herpå.}

Repræsentativiteten i de videnskabelige artikler og studier kan diskuteres, da undersøgelserne primært har inkluderet små populationer og/eller få deltagere. Det systematiske review Virtual Visits in Home Health Care for Older Adults inkluderede 12 artikler, hvoraf antallet af deltagere i de respektive artikler maksimalt bestod af 218 patienter\cite{Baf2}. Desuden inkluderede pilotstudiet Videophone Delivery of Medication Management in Community Nursing kun ni borgere\cite{wade}. ‘Pilotprojekt Videokommunikation’ fra Sundhedscenter Hadsten inkludere fire borgere og to sygeplejersker \textbf{(kilde: evaluering af skærmopkald)}. I ”Evaluering og dokumentation af telesundhed i kommunal hjemmepleje/sygepleje” fra Viborg Kommune indgik i alt 32 borgere\cite{kandidat}. Formålet med denne mini-MTV har dog ikke været at opnå et højt repræsentativt resultat, men at give indsigt i betydningen af virtuel hjemmepleje i Favrskov Kommune, hvorfor de inkluderede studier har været anvendelige. 

Relationer mellem patient og sundhedsprofessionel ændres ved anvendelse af videoopkald i virtuel hjemmepleje sammenlignet med fysiske hjemmeplejebesøg. Kommunikative forhold mellem borgeren og den sundhedsprofessionelle forandres, idet dialogen ikke længere er af fysisk karakter, men virtuel. Umiddelbart tyder resultater på, at kommunikationen mellem borger og sundhedsprofessionel via videoopkald har været tilfredsstillende\cite{Baf2}, \textbf{Interviewundersøgelse}). Dog foretrak nogle borgere fysiske besøg fremfor virtuelle hjemmeplejebesøg, idet de fysiske besøg var mere naturlige\cite{kandidat}. Borgernes forventninger til hjemmeplejebesøg – hvad enten det var fysisk eller virtuel hjemmepleje – har formodentligt influeret på tilfredsheden af kommunikationen, da nogle borgere formentlig har fundet det tilfredsstillende, at det virtuelle hjemmeplejebesøg har været mere koncentreret om den pågældende ydelse, mens andre borgere har fundet det mere tilfredsstillende, at kommunikationen også har involveret andre aspekter af borgerens liv.

Implementering og brug af virtuel hjemmepleje har affødt flere bekymringer, hvoraf en essentiel bekymring er, at 

\begin{center}
\textit{“the essence of nursing is contact and engagement with people, which involves physical closeness, intimacy, and interpersonal sharing and caring that cannot be approached with computer technology”\cite{telenursing} \textbf{siddetal}.}
\end{center}

Resultater viser dog, at det mere er en bekymring blandt sundhedsprofessionelle end blandt borgere og patienter\cite{telenursing}, \textbf{flere}. Borgere og patienter oplever en stor tryghed ved levering af virtuel hjemmepleje, hvilket understøtter, at bekymringen ved anvendelse af virtuel hjemmepleje primært stammer fra de sundhedsprofessionelle\cite{Baf2}, \textbf{Interviewundersøgelse}. 

Accepten af virtuel hjemmepleje er afhængig af borgerens oplevelser af frihed i forbindelse med virtuel hjemmepleje. En væsentlig parameter, der kan influere på borgerens oplevelse af frihed, er ventetiden ved levering af hjemmepleje. Ventetiden influerer på både individuelle og sociale forhold, for eksempel borgerens familieliv, fritidsliv og arbejdsliv. Oplevelsen af frihed forbundet med virtuel hjemmepleje synes at være blandet. Sammenlignet med fysisk hjemmepleje fandt nogle borgere stor tilfredshed med videoopkald, da disse kunne planlægges hensigtsmæssigt i forhold til den pågældende borgers hverdag og daglige aktiviteter\cite{kandidat},\cite{wade}. Dog var det vigtigt for borgerne, at videoopkaldene var planlagt på faste tidspunkter, så borgerne kunne opretholde daglige aktiviteter og gøremål\cite{kandidat}. \textbf{Mere}

\section{Konklusion}
På baggrund af resultat- og diskussionsafsnittet kan det konkluderes, at tilfredsheden ved implementering og drift af virtuel hjemmepleje med videokonference i Favrskov Kommune har været høj. Det kan ud fra interviewundersøgelsen fra ’Pilotprojekt Videokommunikation’ konkluderes, at der var en høj tilfredshed blandt de deltagende borgere. Denne tilfredshed understøttes yderligere af de konkrete studier, der er medtaget i denne MTV.

Det kan endvidere konkluderes, at borgeraccepten af virtuel hjemmepleje med videokonference i Favrskov Kommune har været stor. Borgerne i interviewundersøgelsen har været positive og betydningen af virtuel hjemmepleje har medført positive reaktioner. Virtuel hjemmepleje har medvirket til en følelse af frihed blandt borgerne i Favrskov Kommune.

Yderligere kan det konkluderes, at virtuel hjemmepleje med videokonference har skabt en stor tryghed for borgerne i Favrskov Kommune, idet kommunikationen mellem den sundhedsprofessionelle og borgeren har været visuel. 

Samlet kan det konkluderes, at implementering og drift af virtuel hjemmepleje med videokonference i Favrskov Kommune har betydet høj tilfredshed samt stor borgeraccept og har ligeledes betydet en stor følelse af tryghed for borgerne. 












