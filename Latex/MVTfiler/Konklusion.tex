\chapter{Konklusion}

Det konkluderes, at Appinux’ løsning til virtuel hjemmepleje kan erstatte eller supplere fysiske besøg med en acceptabel billedkvalitet uden forsinkelser på lyd, såfremt dækning og båndbredde er tilstrækkelig i Favrskov Kommune. Det kan endvidere konkluderes, at det er fordelagtigt at undersøge dækningsforholdene ude ved borgerne, før løsningen implementeres for at undgå tekniske problemer.
Yderligere kan det konkluderes, at Appinux overholder minimumskravene i forhold til datakryptering. Det kan ikke konkluderes, hvorvidt Appinux og Favrskov Kommune overholder sikkerhedskravene vedrørende login til Appinux' løsning, der kommunikerer via det åbne internet. 
At Favrskov Kommune på forhånd er bekendt med mulighederne for at hive data ud af systemet og desuden er bekendt med mulighederne for at bygge videre på løsningen, så et eventuelt leverandørskifte kan forekomme uden problemer, er fordelagtig. På baggrund heraf kan det konkluderes, at det er vigtigt, at kompatibiliteten på forhånd er undersøgt.

Accepten og tilfredsheden blandt borgere i Favrskov Kommune har været positiv, og på baggrund heraf kan det konkluderes, at betydningen af implementering af virtuel hjemmepleje med videoopkald har været positiv. Desuden kan det konkluderes, at trygheden ved videoopkald er stor. Dette underbygges yderligere af de studier, der er medinddraget.

Ud fra et organisatorisk aspekt kan det konkluderes, at arbejdsgangene i Favrskov Kommune er ændret, og at der er sket ændringer i konkrete arbejdsteknikker. 
Det kan ligeledes konkluderes, at den organisatoriske implementering af Appinux' løsning med videoopkald har været succesfuld, og at reaktionen fra de sundhedsprofessionelle primært har været positiv. 

Ved implementering af Appinux' løsning med videoopkald kan det konkluderes, at opstartomkostningerne er store, hvorfor en kortsigtet økonomisk opgørelse er negativ.
Mange variabler influerer på de økonomiske resultater, og det er vigtigt at identificere samtlige variabler, før et samlet økonomisk resultat kan udregnes. Dog kan det konkluderes, at der på lang sigt er potentiale for økonomiske besparelser ved implementering af Appinux' løsning med videoopkald i Favrskov Kommune.  

Samlet kan det konkluderes, at Appinux’ løsning med videoopkald i Favrskov Kommune tildels imødekommer de teknologiske forudsætninger. De borgermæssige rammer for tilfredshed, borgeraccept og tryghed er opfyldt og organisatorisk har modtagelsen af ændringer i arbejdsgange være overvejende positiv. Økonomisk set er der ikke et entydigt svar på, hvorvidt implementering af videoopkald i Favrskov Kommune er rentabelt.






