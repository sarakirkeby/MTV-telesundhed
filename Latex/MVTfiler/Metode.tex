\chapter{Metode}\label{chap:metode}

\section{Litteraturstudie}
Denne mini-MTV's data og informationer er indhentet gennem litteraturstudier. Videnskabelig litteratur omhandlende videobaserede telesundhedsløsninger for hjemmepleje er søgt på følgende databaser: PubMed, Embase, CINAHL, Cochrane Library, Engineering Village og Google Scholar. Litteratursøgningsprocessen er udvidet til også at inkludere artikler identificeret ved kædesøgning i referencelister. Søgeprotokoller, kædesøgning, emneord, eksklusions- og inklusionskriterier er vedlagt som bilag [Bilag 16, 16.1]. 

På baggrund af inklusions- og eksklusionskriterierne er antallet af artikler inkluderet i denne mini-MTV lig med 13. Størstedelen af artiklerne er udenlandske, men er vurderet repræsentative for denne mini-MTV, idet parametrene, som undersøges er sammenlignelige. En fuldstædig generalisering er ikke muligt, idet sundhedsforholdene varierer i de forskellige lande.   

\section{Generel dataindsamling}
Data er endvidere indhentet gennem møder med forskellige interessenter: Appinux, Netplan Care og medarbejdere i Favrskov Kommune [Bilag 6, 6.1], [Bilag 4, 4.1]. 

\subsection{Empirisk dataindsamling}
Med baggrund i de fokuserede spørgsmål har et stort fokus været at belyse borgernes og de sundhedsprofessionelle oplevelser og erfaringer med virtuel hjemmepleje. Det har derfor været nærliggende at supplere litteraturstudiet og den generelle dataindsamling med en kvalitativ interviewundersøgelse for netop at opnå en indgående og detaljeret viden herom.\\
I forbindelse med evalueringen af \textit{Pilotprojekt Videokommunikation} blev der af Sundhedscenter Hadsten gennemført en lille kvalitativ evalueringsundersøgelse i form af strukturerede interviews med fire borgere og to sygeplejersker [Bilag 7, 7.1]. Data fra denne interviewundersøgelse er indhentet og kritisk vurderet med henblik på anvendelse som empirisk datagrundlag i denne mini-MTV fremfor at igangsætte en ny empirisk videns indsamling. Gyldigheden af denne er diskuteret og vedlagt som bilag [Bilag 16, 16.1].


  
	 

