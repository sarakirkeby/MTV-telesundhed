\chapter{Organisation}

\section{Indledning}
I dette afsnit fokuseres på den organisatoriske betydning af virtuel hjemmepleje i \textit{Pilotprojekt Videokommunikation} med udgangspunkt i arbejdsgangen på Sundhedscenter Hadsten. 

Der er på baggrund af samtaler med kommunen udviklet en case, som beskriver en typisk situation, hvor der bruges henholdsvis virtuel og fysisk hjemmepleje. Det er ud fra denne case, at dele af analysen vil tage udgangspunkt.

\subsection{Fokuseret spørgsmål}

\begin{itemize}
	\item Hvilke organisatoriske betydninger er der ved implementering og drift af virtuel hjemmepleje med videokonference sammenlignet med konventionel fysisk hjemmepleje i Favrskov Kommune? \\Spørgsmålet søges besvaret med udgangspunkt i følgende punkter:
	\begin{itemize}
	\item Forskel i medarbejdernes arbejdsgange før/efter virtuel hjemmepleje
	\item Implementering
	\item De sundhedsprofessionelles reaktion
\end{itemize}
\end{itemize}


\section{Metode}
Informationer i dette afsnit baseret på data indsamlet ved hjælp af møder og emailkorrespondancer med repræsentanter fra Favrskov Kommune. Disse informationer er understøttet af data fra en litteratursøgning i videnskabelige databaser. For en dybdegående beskrivelse af metoden henvises til afsnittet Metode.\textbf{(Jæææææbe)} 

Specifikke emneord: \textit{Caregivers, Telenursing, Telemedicine, Workplace, Videocall.}

\section{Resultater}


\subsection{Ændringer i arbejdsgange}
Sundhedscenter Hadsten hører under hjemmeplejen og ligger under organisationens østlige afdeling [Bilag 12, 12.1]. Fra et organisatorisk aspekt er \textit{Pilotprojekt Videokommunikation} implementeret med henblik på, at der skal spares minutter i sygeplejernes arbejdsdag [Bilag 8, 8.3]. Dette skal give plads til andre opgaver og derved mulighed for økonomiske besparelser i kommunen. 

De sundhedsprofessionelle i hjemmeplejen er overordnet delt i fire teams med to teams i hvert distrikt. Der arbejdes i teams med op til 25 sundhedsprofessionelle i hvert. Ud af disse er tre sundhedsprofessionelle i hver gruppe ansvarlig for videoopkald på nuværende tidspunkt [Bilag 11, 11.45].

Formålet med implementeringen af Appinux’ løsning er en alternativ levering af ydelsen Medicingivning.  Medicingivning er et tilbud, som hjælper borgere med at indtage deres medicin. Denne opgave afhænger af at der foregående er sket medicinadministration. Medicinadministation skal foretages af en sygeplejerske frem for en hjemmeplejer og består i at dele medicinen op i korrekte doseringer til hver dag i ugen [Bilag 4, 4.1].


\subsubsection{Case}
Casen er blevet udviklet på baggrund af informationer indhentet fra Sundhedscenter Hadsten [Bilag 4, 4.1], og beskriver hvordan Medicingivning foregår i det fysiske og virtuelle hjemmeplejebesøg.
Formålet med casen er at belyse ændringerne i arbejdsgangene fra virtuelt til fysisk hjemmeplejebesøg.

\begin{table}[H]
	\caption{Case for Medicingivning. Tabellen viser den typiske arbejdsgang i forbindelse med et fysisk besøg og et virtuelt besøg; forbederelse til, under, efter besøget. Antallet af fysiske besøg pr. borger pr. dag kan maksimalt være fire, hvoraf maksimalt to kan erstattes af virtuelle besøg.}
	\centering
	\label{tab:Case}
	\begin{tabularx}{\textwidth}{|l|X|X|}
		
		\hline
		  & \textbf{Fysisk besøg} & \textbf{Virtuelt besøg}\\ \hline
		& & \\Forberedelse & Der aftales mellem borger og sundhedsprofessionelle hvordan mødet skal foregå. & Der aftales mellem borger og sundhedsprofessionelle hvornår mødet skal være.\\ & &\\
		 & Borgeren bliver skrevet på kørelisten i forhold til aftale. & Borgeren bliver skrevet på den separat køreliste til videoopkald.\\  & &\\
		 & Medicinen er opdelt i ugebokse på forhånd. & Medicinen er opdelt i ugebokse på forhånd.\\[4ex] \hline & & \\
		 Under & Den sundhedsprofessionelle kører ud til borger. & Den sundhedsprofessionelle ringer op.\\  & &\\
		  & Den sundhedsprofessionelle hjælper borgeren med at tage sin medicin. & Borgeren besvarer opkaldet og bliver guidet igennem medicintagningen.\\ & &\\
		  & Den sundhedsprofessionelle forlader borger. & Den sundhedsprofesionelle vurderer arbejdet som gjort og afslutter opkaldet.\\[4ex] \hline & & \\Efter & Opgaven bliver vinget af som gennemført. & Opgaven bliver vinget af som gennemført.\\[4ex] \hline
		  & & \\Hændelser & Op til 4 gange i døgnet & 1-2 gange i døgnet.\\ [4ex] \hline
		  & & \\Tid & 10 minutter. & 2-3 minutter.\\[4ex] \hline
	\end{tabularx}
\end{table}

Som det fremgår af ovenstående case \ref{tab:Case} er de største ændringer af arbejdsgangene tidsbesparelser, og at den sundhedsprofessionelle kan udføre ydelsen Medicingivning på farten. Med Appinux' løsning sidder de sundhedsprofessionelle foran en tablet og ringer borgeren op. Borgeren er ved hjælp af en tablet i deres eget hjem i stand til at høre og se den sundhedsprofessionelle og kan derved blive guidet igennem Medicingivning [Bilag 4, 4.1]. For en teknisk beskrivelse af Appinux' løsning se da [\textbf{jæææææææbe}].


\subsubsection{Nye arbejdsgange}
\textit{\textbf{Call-center}}\\
Der er etableret et call-center i Sundhedscenter Hadsten, hvor de sundhedsprofessionelle kan foretage opringningerne til borgerne [Bilag 11, 11.28]. Videoopkaldene kan enten foretages i dette call-center eller på en tablet. Call-centeret var etableret før, videoopkald kom på tale i kommunen og blev da brugt som et regulært call-center [Bilag 4, 4.1].

De sundhedsprofessionelle holder øje med observationsoverblikket i Appinux for at se, hvem der skal ringes til, og om en borger har ringet til centralen uden at have en aftale. Der er op til tre sundhedsprofessionelle, som har ansvaret for videoopkald fra call-centeret, og der aftales internt i denne gruppe, hvem der er designeret de forskellige opkald [Bilag 4, 4.1].

\textit{\textbf{Superbrugere}}\\
I forbindelse med at systemet blev implementeret, blev der oprettet superbrugerroller, som blev pådraget enkelte sundhedsprofessionelle. Disse superbrugere har det overordnede ansvar omkring applikationen. De sørger for oprette nye borgere i systemet samt at slette borgere i tilfælde af ændret behov. Superbrugerne sørger også for at opdatere ændringer, såsom adresseskift og reorganisering af teams [Bilag 8, 8.4].

Der er udarbejdet vejledninger til superbrugerne, som detaljeret beskriver fremgangsmetoden til de forskellige scenarier [Bilag 13, 13.1]. Hvert halve år afholder de et fælles superbrugermøde af halvanden times varighed for alle superbrugererne [Bilag 4, 4.1]. Ud over disse superbrugere er der to hovedansvarlige for \textit{Pilotprojekt Videokommunikation}.

\textit{\textbf{Opdatering og support}}\\
Sundhedscenter Hadsten er ansvarlige for at opdatere systemet hver gang Appinux stiller en opdatering til rådighed. Der kan forekomme fire opdateringer årligt, men Sundhedscenter Hadsten har mulighed for at fravælge hver anden opdatering og kun opdatere systemet én gang hvert halve år.
Det er ligeledes call-centeret i Sundhedscenter Hadsten, der i første omgang er ansvarlig for support [Bilag 4, 4.1]. 

\subsection{Implementering}
Virtuel hjemmepleje blev afprøvet i Favrskov Kommune i form af et \textit{Pilotprojekt Videokommunikation}, som blev udført i starten af 2015. Det primære ansvar for implementering af teknologien, har ligget hos Sundhedscenter Hadsten. Implementeringen har således ikke været drevet af Appinux, som dog har givet indledende support om blandt andet valg af udstyr [Bilag 4, 4.1].

Favrskov Kommune forventer, at \textit{Pilotprojekt Videokommunikation} er færdigimplementeret i alle kommunens fire distrikter i maj 2016 [Bilag 4, 4.1]. I forbindelse med den fulde implementering, forventes det at videoopkald kommer til at blive brugt af samtlige sundhedsprofessionelle i de fire teams. Kommunen forventer, at visitationen kan overtage projektet fra oktober og derved kan tilbyde videoopkald i stedet for fysiske besøg til borgere, som er egnede [Bilag 4, 4.1].

De to hovedansvarlige for \textit{Pilotprojekt Videokommunikation} har undervist til de øvrige sundhedsprofessionelle, som har deltaget i pilotprojektet. Desuden blev der ved opstart af pilotprojektet udleveret vejledninger til de sundhedsprofessionelle, som beskriver opkaldsforløbet til borgerne. Denne vejledning kombinerer tekst med billeder for en lettere forståelse. Denne vejleder samtidig om "god skærmopførsel"\ for at sikre trygheden for borgeren [Bilag 13, 13.1].  

Denne vejledning har vist sig brugbar for de sundhedsprofessionelle. De vil dog gerne have en mere detaljeret vejledning og efterspørger generelt undervisning i systemet [Bilag 7, 7.1].

 
\subsection{De sundhedsprofessionelles reaktion}
\textit{Pilotprojektet Videokommunikation} blev pålagt de deltagende teams som et virtuelt alternativ til fysisk hjemmeplejebesøg ved Medicingivning. 

I interviewundersøgelsen fra \textit{Pilotprojekt Videokommunikation} gav to sygeplejesker udtryk for at have blandede præferencer omkring systemet. Interviewundersøgelsen fortæller også, at sundhedsprofessionelle har reflekteret over, hvordan videoopkald kunne videreudvikles i kommunen [Bilag 7, 7.1]. 

Der har været positiv respons fra mange af de sundhedsprofessionelle omkring pilotprojektet. Responsen påpeger primært, hvor meget tid det er muligt at spare \textbf{kilde}.

De primære negative præferencer omkring pilotprojektet omhandler de tekniske problemer, som de sundhedsprofessionelle oplevelse i forbindelse med systemet i form af forsinkelser med lyd og forringet billedkvalitet. 

Efter pilotprojektets opstart har der været et møde omkring projektet, hvor involverede sundhedsprofessionelle og projektansvarlige har evalueret på projektets udførelse. På dette møde blev der nævnt de samme tekniske problemer, som der tidligere var hørt fra sundhedsprofessionelle. Der blev desuden fastlagt procedurer i tilfælde af, at der ikke kunne opnå kontakt med borgeren [Bilag 8, 8.2]. 

Det har ikke været muligt at indsamle fyldestgørende information vedrørende beslutningstagningen for indførelse af pilotprojektet.

  
\section{Diskussion}
\subsubsection{Organisationens forudsætninger}
Diskussionen vil tage udgangspunkt i artiklen \citetitle{Ikt}. Denne artikel deler implementeringen af en ny teknologi op i tre dele. Modellens tre aspekter består af organisationens forudsætninger, medindragelse af systemets brugere og den differentierede udvikling af brugere. Disse tre aspekter kan spejles til forudsætninger før, under og efter implenteringen \cite{Ikt}. 

De tre paragraffer af diskussionen vil repræsentere \textit{Pilotprojekt Videokommunikation} i Favrskov Kommune i forhold til disse aspekter. 

Appinux’ løsning blev ikke introduceret til Favrskov Kommune som en løsning på et problem, men i stedet som en ændring i arbejdsteknik. Der er ikke udført en indledende undersøgelse på traditionel vis, da Appinux' løsning oprindeligt blev indført med et anden modul og på baggrund heraf er senere udvidet med modulet til videoopkald [Bilag 4, 4.1].

Pilotprojektet er forsøgt præsenteret på en måde, så det så vidt muligt ligner de sundhedsprofessionelles tideligere arbejdsgange. I de relevante organisatoriske afdelinger er der uddannet superbrugere, udarbejdet definition af ansvarsområder og fastlæggelser af scenarier \textbf{Denne del kan muligvis slettes}. 

\subsubsection{Medinddragelse af borgere og sundhedsprofessionelle}
I introduktionsfasen blev pilotprojektet modtaget med blandede præferencer [Bilag 11, 11.45]. Dette kan underbygges af mangel på introduktion hos de sundhedsprofessionelle, som organisationsmodellen i \citetitle{Ikt} foreslår. 

Igennem implementeringsfasen har de sundhedsprofessionelles, trods mærkbare tekniske problemer, erfaringer med virtuelle besøg primært været positive. De sundhedsprofessionelle mener, at deres arbejde bliver nemmere ved hjælp af denne teknik. Der har altså ikke været nogen frygt for, at de virtuelle besøg kan erstatte nuværende arbejdsopgaver. I følge det randomiserede forsøg \citetitle{telenursing} giver de sundhedsprofessionelle udtryk for bekymring for, at de virtuelle hjemmeplejebesøg på længere sigt vil formindske antallet af arbejdspladser. Tværtimod har reaktionen været overvejende positiv, og flere af de sundhedsprofessionelle kan se mulighederne i virtuel hjemmepleje og har selv kommet med forslag til udvidelser af systemet [Bilag 4, 4.1], [Bilag 11, 11.1]. Det virker som om, at organisationen er vokset til at se systemet som et hjælpemiddel frem for en erstatning.

Ifølge \citetitle{Ikt} er støtte fra ledelsens side en vigtig faktor for succesfuld implementering af informations- og kommunikationsteknologier \cite{Ikt}. Det har ikke været muligt at indsamle fyldestgørende information om ledelsens støtte til \textit{Pilotprojekt Videokommunikation} i Favrskov Kommune. 

\subsubsection{Differentieret uddannelse af borger og sundhedsprofessionelle}
For at et system kan fungere optimalt kræver det, at brugerne har forudsætninger for at bruge systemet optimalt. Dette kræver undervisning \cite{Ikt}. I Favrskov Kommune har der været to hovedansvarlige, som sørger for undervisning af Appinux' løsning. 

Arbejdsgangene ved virtuel hjemmeplejebesøg i Favrskov Kommune har ændret sig marginalt i forhold til arbejdsgangene ved fysisk besøg. Ændringen er ikke en ændring i arbejdsgang, men en ændring i arbejdsteknik, idet de sundhedsprofessionelle sidder foran en tablet, og udfører det arbejde, som de ellers ville have udført i borgerens eget hjem \textbf{Måske dette ikke behøves?}. Dette stemmer overens med, at videoopkald i artikel \citetitle{telenursing} ikke anses som en ændring i en arbejdsgang, men i stedet som en ny arbejdsteknik \cite{telenursing}.  

Et problem som de sundhedsprofessionelle hurtigt påpegede, var problemerne med at borgere måske ikke var kompatible med den ydelse der kunne tilbydes nu. En bekymring var at de ældre, de tager sig af, måske ikke har de mentale eller tekniske egenskaber, der skal til, for at kunne udføre et videomøde. Dette er dog imødekommet ved at starte ud småt, og kun tilbyde støtte til medicintagnings- og måltidshjælp. Problemer kan dog stadig forekomme i takt med at ydelsen bliver udvidet, og flere kan blive egnet. Dette kan for eksempel være, at nogle ældre som er egnet til at modtage ydelsen, har deres medicin låst væk. På nuværende tidspunkt har Favrskov kommune ingen måde at kunne låse en medicin op digitalt, og derved vil videoopkald ikke være egnet i den sammenhæng [Bilag 4, 4.1]. Desuden vil visitationen altid foretrække at give ældre rehabiliteringskurser, frem for at tilbyde ydelser. Med medicingivning giver du ydelser, og ikke selvoptræning. 
\textbf{Kig på dette... skal det nævnes i resultater også.}


\section{Delkonklusion}
Der konkluderes fra resultats- og diskussionsafsnittet at forskellene i arbejdsgange er små, da videoopkald er blevet implemteret som en ny arbejdsteknik, og ikke en ny arbejdsopgave. Der er vidreudviklet på eksisterende superbrugere og andre roller, aom har organisatiorisk relevans for projetet. Der er desuden oprettet supportroller (Skal måske kaldes noget andet?) i form af hovedansvarlige for projektet. 

Implementeringen af pilotprojektet er forløbet planmæssigt og har været en succes. Implementeringen var selvdrevet af kommunen, og disse har derved selv stået for undervisning og  vejledninger omkring pilotprojektet.

Ud fra interviewresultater fra Favrskov kommune, ses det at de sundhedsprofessionelle har reageret med blandede prefærencer omkring projektet. Modtagelsen har primært været positiv, på trods af mærkbare tekniske problemer. Dette succeskriterie er understøttet af studier, medtaget i dette kapitel. De sundhedsprofessionelle ser desuden fordele ved at udvide opkald til andre aspekter af deres arbejdsdag. 

Den organisatoriske implementering af pilotprojektet vurderes til at være en succes. Favrskov kommune har implementeret pilotprojektet, samtidig med at der er taget højde for ændringen i arbejdsteknik. Pilotprojektet har desuden sikret sig opbakning fra de sundhedsprofessionelle. 
