\chapter{Organisation}

\section{Indledning}
I dette kapitel fokuseres på den organisatoriske betydning af virtuel hjemmepleje i \textit{Pilotprojekt Videokommunikation} med udgangspunkt i arbejdsgangen på Sundhedscenter Hadsten. 

Der er på baggrund af samtaler med kommunen udviklet en case, som beskriver en typisk situation, hvor der bruges henholdsvis virtuel og fysisk hjemmepleje. Dele af analysen tager udgangspunkt i denne case. 

\subsection{Fokuseret spørgsmål}

\begin{itemize}
	\item Hvilke organisatoriske betydninger er der ved implementering og drift af virtuel hjemmepleje med videoopkald sammenlignet med konventionel fysisk hjemmepleje i Favrskov Kommune? \\Spørgsmålet søges besvaret med udgangspunkt i følgende punkter:
	\begin{itemize}
	\item Forskel i arbejdsgange før/efter implementeringen af virtuel hjemmepleje
	\item Implementering
	\item De sundhedsprofessionelles reaktion
\end{itemize}
\end{itemize}


\section{Metode}
Informationer i dette kapitel er baseret på data indsamlet på baggrund af møder og emailkorrespondancer med repræsentanter fra Favrskov Kommune. Disse informationer er understøttet af data fra et litteraturstudie.  For en dybdegående beskrivelse af metoden henvises til bilag [Bilag 16, 16.1].

Specifikke emneord: \textit{Caregivers, Telenursing, Telemedicine, Workplace, Videocall, Telehealth.}

\section{Resultater}
\subsection{Ændringer i arbejdsgange}\label{sec:arbejdsgange}
Sundhedscenter Hadsten hører under hjemmeplejen og ligger under organisationens østlige distrikt [Bilag 12, 12.1]. Fra et organisatorisk aspekt er \textit{Pilotprojekt Videokommunikation} implementeret med henblik på at spare minutter i de sundhedsprofessionelles arbejdsdag for at give plads til andre arbejdsopgaver og derved mulighed for økonomiske besparelser [Bilag 8, 8.3]. 

De sundhedsprofessionelle i hjemmeplejen er opdelt i fire teams med to i hvert distrikt med op til 25 sundhedsprofessionelle i hvert. Ud af disse er tre sundhedsprofessionelle i hvert team ansvarlige for videoopkald [Bilag 11, 11.45].

Formålet med implementeringen af Appinux’ løsning er en alternativ levering af ydelsen Medicingivning. Medicingivning er et tilbud, som hjælper borgere med at indtage sin medicin. Denne opgave afhænger af, at der foregående er sket Medicinadministration. Medicinadministration skal foretages af en sygeplejerske frem for en hjemmeplejer og består i at dele medicinen op i korrekte doseringer til hver dag i ugen [Bilag 4, 4.1].


\subsubsection{Case}
Casen er blevet udviklet på baggrund af informationer indhentet fra Sundhedscenter Hadsten og beskriver, hvordan Medicingivning foregår i det fysiske og virtuelle hjemmeplejebesøg [Bilag 4, 4.1].
Formålet med casen er at belyse ændringerne i arbejdsgangene fra virtuelt til fysiske besøg.

\begin{table}[H]
	\caption{Case for Medicingivning. Tabellen viser den typiske arbejdsgang i forbindelse med et fysisk og et virtuelt besøg; forbederelse til, under og efter besøget. Antallet af fysiske besøg pr. borger pr. dag kan maksimalt være fire, hvoraf maksimalt to kan erstattes af virtuelle besøg.}
	\centering
	\label{tab:Case}
	\begin{tabularx}{\textwidth}{|l|X|X|}
		
		\hline
		  & \textbf{Fysisk besøg} & \textbf{Virtuelt besøg}\\ \hline
		& & \\\textbf{Forberedelse} & Der aftales mellem borger og sundhedsprofessionelle hvordan mødet skal foregå. & Der aftales mellem borger og sundhedsprofessionelle hvornår mødet skal være.\\ & &\\
		 & Borgeren bliver skrevet på kørelisten i forhold til aftale. & Borgeren bliver skrevet på den separat køreliste til videoopkald.\\  & &\\
		 & Medicinen er opdelt i ugebokse på forhånd. & Medicinen er opdelt i ugebokse på forhånd.\\[4ex] \hline & & \\
		\textbf{ Under} & Den sundhedsprofessionelle kører ud til borger. & Den sundhedsprofessionelle ringer op.\\  & &\\
		  & Den sundhedsprofessionelle hjælper borgeren med at tage sin medicin. & Borgeren besvarer opkaldet og bliver guidet igennem medicintagningen.\\ & &\\
		  & Den sundhedsprofessionelle forlader borger. & Den sundhedsprofesionelle vurderer arbejdet som gjort og afslutter opkaldet.\\[4ex] \hline & & \\\textbf{Efter} & Opgaven bliver vinget af som gennemført. & Opgaven bliver vinget af som gennemført.\\[4ex] \hline
		  & & \\\textbf{Hændelser} & Op til 4 gange i døgnet & 1-2 gange i døgnet.\\ [4ex] \hline
		  & & \\\textbf{Tid} & 10 minutter. & 2-3 minutter.\\[4ex] \hline
	\end{tabularx}
\end{table}

Som det fremgår af ovenstående case tabel \ref{tab:Case} er de største ændringer af arbejdsgangene tidsbesparelser, og at den sundhedsprofessionelle kan udføre ydelsen Medicingivning fra call-centeret eller på farten. Med Appinux' løsning sidder de sundhedsprofessionelle foran en tablet og ringer borgeren op. Borgeren er ved hjælp af en tablet i deres eget hjem i stand til at høre og se den sundhedsprofessionelle og kan derved blive vejledt igennem Medicingivning [Bilag 4, 4.1]. For en teknisk beskrivelse af Appinux' løsning henvises til \vref{chap:teknologiafsnit}.


\subsubsection{Nye arbejdsgange}
Udover de nye arbejdsgange skitserede i casen tabel \ref{tab:Case} er yderligere følgende nye arbejdsgange identificeret. 

\textit{\textbf{Call-center}}\\
Der er etableret et call-center i Sundhedscenter Hadsten, hvor de sundhedsprofessionelle kan foretage opringningerne til borgerne [Bilag 11, 11.28]. Call-centeret var etableret før, videoopkaldsmodulet kom på tale i kommunen og blev da brugt som et regulært call-center [Bilag 4, 4.1].

De sundhedsprofessionelle holder øje med observationsoverblikket\footnote{Observationsoverblikket er et overblik over, hvilke borgere, der skal kaldes op samt hvilke borgere, der har foretaget indgående opkald uden aftale [Bilag 4, 4.1].} i Appinux' platform. Der er op til tre sundhedsprofessionelle, som har ansvaret for videoopkald fra call-centeret, og der aftales internt blandt disse tre, hvem der er designeret de forskellige opkald [Bilag 4, 4.1].

\textit{\textbf{Superbrugere}}\\ 
I forbindelse med implementeringen af Appinux' løsning, blev der oprettet superbrugerroller, som blev pådraget enkelte sundhedsprofessionelle. Disse superbrugere har det overordnede ansvar omkring løsningen. De sørger for oprette nye borgere i systemet samt at slette borgere i tilfælde af ændret behov. Superbrugerne sørger også for at opdatere ændringer såsom adresseskift og reorganisering af teams [Bilag 8, 8.4].

Der er udarbejdet vejledninger til superbrugerne, som detaljeret beskriver fremgangsmetoden af forskellige scenarier [Bilag 13, 13.1]. Hvert halve år afholdes et fælles superbrugermøde af halvanden times varighed [Bilag 4, 4.1]. Udover disse superbrugere er der to hovedansvarlige for \textit{Pilotprojekt Videokommunikation}.

\textit{\textbf{Opdatering og support}}\\
Sundhedscenter Hadsten er ansvarlig for at opdatere Appinux' løsning, hver gang Appinux stiller en opdatering til rådighed. 
Det er ligeledes de to hovedansvarlige for \textit{Pilotprojekt Videokommunikation} i Sundhedscenter Hadsten, der i første omgang er ansvarlige for support [Bilag 4, 4.1]. 

\subsection{Implementering}
Virtuel hjemmepleje blev afprøvet i Favrskov Kommune i form af \textit{Pilotprojekt Videokommunikation}, som blev udført i starten af 2015. Det primære ansvar for implementering af teknologien har ligget hos Sundhedscenter Hadsten. Implementeringen har således ikke været drevet af Appinux, som dog har givet indledende support i form af konfiguration og uddannelse af superbrugere [Bilag 4, 4.1].

Favrskov Kommune forventer, at \textit{Pilotprojekt Videokommunikation} er færdigimplementeret i alle kommunens fire distrikter i maj 2016 [Bilag 4, 4.1]. I forbindelse med den fulde implementering forventes det, at videoopkald kommer til at blive brugt af samtlige sundhedsprofessionelle i hjemmeplejens fire teams. Kommunen forventer, at visitationen kan overtage projektet fra oktober 2016 og derved tilbyde videoopkald i stedet for fysiske besøg til visiterede borgere [Bilag 4, 4.1].

De to hovedansvarlige for \textit{Pilotprojekt Videokommunikation} har undervist de øvrige sundhedsprofessionelle, som har deltaget i pilotprojektet. Desuden blev der ved opstart af pilotprojektet udleveret vejledninger til de sundhedsprofessionelle, som beskriver opkaldsforløbet til borgerne. Denne vejledning kombinerer tekst med billeder for en lettere forståelse og vejleder samtidig om "god skærmopførsel"\ for at sikre trygheden for borgeren [Bilag 13, 13.1].  

Denne vejledning har vist sig brugbar for de sundhedsprofessionelle, som dog efterspørger en mere detaljeret vejledning og generel undervisning i Appinux' løsning [Bilag 7, 7.1].

 
\subsection{De sundhedsprofessionelles reaktion}
\textit{Pilotprojekt Videokommunikation} blev pålagt de deltagende teams som et virtuelt alternativ til fysisk hjemmeplejebesøg ved Medicingivning. 

I interviewundersøgelsen fra \textit{Pilotprojekt Videokommunikation} gav to sygeplejesker udtryk for at have blandede holdninger om Appinux' løsning. Interviewundersøgelsen fortæller også, at de sundhedsprofessionelle har reflekteret over, hvordan videoopkald kan videreudvikles i kommunen [Bilag 7, 7.1]. 

Der har været positiv respons fra de sundhedsprofessionelle omkring pilotprojektet. Responsen påpeger primært, at tidsbesparelser ved virtuel hjemmepleje er mulige [Bilag 7, 7.1].

De negative holdninger om pilotprojektet omhandler de tekniske problemer, som de sundhedsprofessionelle oplevede i forbindelse med løsningen i form af forsinkelser med lyd og forringet billedkvalitet. I interviewundersøgelsen påpegede de sundhedsprofessionelle endvidere bekymring om, at nogle borgere muligvis ikke er egnede til at modtage Medicingivning via videoopkald på baggrund af manglende mentale eller tekniske egenskaber på trods af, at de er visiteret hertil [Bilag 7, 7.1].   

Efter pilotprojektets opstart er et evalueringsmøde blevet afholdt. På dette møde blev de samme tekniske problemer nævnt, som de sundhedsprofessionelle også påpegede tidligere i interviewundersøgelsen. Der blev desuden fastlagt procedurer i tilfælde af, at der ikke kunne opnås kontakt med borgeren [Bilag 8, 8.2]. 

Det har ikke været muligt at indsamle fyldestgørende information vedrørende beslutningstagningen for indførelse af pilotprojektet.

  
\section{Diskussion}
\subsubsection{Organisationens forudsætninger}
Artiklen \citetitle{Ikt} deler implementeringen af en ny teknologi op i tre dele. Modellens tre aspekter består af organisationens forudsætninger, medinddragelse af systemets brugere og den differentierede udvikling af brugere. Disse tre aspekter kan spejles til forudsætninger før, under og efter implementeringen \cite{Ikt}. De tre afsnit af diskussionen vil repræsentere \textit{Pilotprojekt Videokommunikation} i Favrskov Kommune i forhold til denne artikel. 

Appinux’ løsning blev ikke introduceret til Favrskov Kommune som en løsning på et problem, men i stedet som en ændring i arbejdsgang. Der er ikke udført en indledende undersøgelse på traditionel vis, da Appinux' løsning oprindeligt blev indført med et andet modul og er på baggrund heraf senere udvidet med modulet til videoopkald [Bilag 4, 4.1]. 

\subsubsection{Medinddragelse af borgere og sundhedsprofessionelle}
I introduktionsfasen var modtagelsen af \textit{Pilotprojekt Videokommunikation} blandet [Bilag 11, 11.45]. Dette kan underbygges af mangel på introduktion hos de sundhedsprofessionelle, som implementeringsmodellen i \citetitle{Ikt} foreslår. 

Igennem implementeringsfasen har de sundhedsprofessionelles, trods tekniske problemer, erfaringer med virtuelle besøg primært været positive. De sundhedsprofessionelle mener, at deres arbejde bliver nemmere ved hjælp af denne teknik. Ifølge det randomiserede studie \citetitle{telenursing} udtrykker de sundhedsprofessionelle bekymring for, om de virtuelle hjemmeplejebesøg på længere sigt vil formindske antallet af arbejdspladser. I kontrast til denne, har reaktionen i Favrskov Kommune været overvejende positiv, og flere af de sundhedsprofessionelle kan se mulighederne i virtuel hjemmepleje og er selv kommet med forslag til udvidelser af Appinux' løsning [Bilag 4, 4.1], [Bilag 7, 7.1]. Det tyder på, at organisationen er vokset til at kunne se løsningen som et hjælpemiddel frem for en erstatning.

Ifølge \citetitle{Ikt} er støtte fra ledelsens side en vigtig faktor for succesfuld implementering af informations- og kommunikationsteknologier \cite{Ikt}. Det har ikke været muligt at indsamle fyldestgørende information om ledelsens støtte til \textit{Pilotprojekt Videokommunikation} i Favrskov Kommune. 

\subsubsection{Differentieret uddannelse af borgere og sundhedsprofessionelle}
For at et system kan fungere optimalt kræver det, at brugerne har forudsætninger for at bruge systemet optimalt, hvilket kræver undervisning \cite{Ikt}. I Favrskov Kommune har de to hovedansvarlige sørget for undervisning i Appinux' løsning til involverede sundhedsprofessionelle. 

I forbindelse med Medicingivning sker ændringen i arbejdsgangen i selve arbejdsteknikken, hvorpå ydelsen udføres. Dette skyldes, at ydelsen, der leveres, er den samme uanset, om den leveres som et fysisk eller virtuelt besøg. Det er derfor en forudsætning, at de sundhedsprofessionelle får den rette uddannelse for at kunne udføre disse nye arbejdsteknikker.  

I interviewundersøgelsen påpegede de sundhedsprofessionelle bekymring om, at nogle borgere muligvis ikke er egnede til at modtage Medicingivning via videoopkald på baggrund af manglende mentale eller tekniske egenskaber [Bilag 7, 7.1]. Det er derfor vigtigt, at visitationen er opmærksom på dette. Desuden vil visitationen altid foretrække at give borgere rehabiliteringskurser, frem for at tilbyde ydelser. Medicingivning er en ydelse og ikke rehabilitering [Bilag 4, 4.1].

\section{Delkonklusion}
På baggrund af resultats- og diskussionsafsnittet kan det konkluderes, at der er sket ændringer i arbejdsgange i form af etablering af et call-center, uddannelse af superbrugere samt udførelse af opdatering og support af Appinux' løsning. Desuden er der sket en ændring i måden, hvorpå ydelsen Medicingivning leveres.

Den organisatoriske implementering af \textit{Pilotprojekt Videokommunikaiton} er forløbet planmæssigt og har været en succes. Implementeringen var selvdrevet af kommunen, som selv har stået for undervisning og vejledning omkring pilotprojektet.

På baggrund af interviewundersøgelsen fra \textit{Pilotprojekt Videokommunikation} har modtagelsen af Appinux' løsning fra de sundhedsprofessionelle været blandet, men modtagelsen har været overvejende positiv, på trods af tekniske problemer. Dette succeskriterie er understøttet af studier medtaget i dette kapitel. De sundhedsprofessionelle ser desuden fordele ved at udvide løsningen til andre funktioner af deres arbejdsdag. 

Den organisatoriske implementering af pilotprojektet vurderes til at være en succes. Favrskov kommune har implementeret pilotprojektet, samtidig med at der tages højde for ændringen i arbejdsgang. Pilotprojektet har desuden sikret sig opbakning fra de sundhedsprofessionelle.

Den organisatoriske betydning af implementering og drift af virtuel hjemmepleje med videoopkald i Favrskov Kommune har betydet nye arbejdsgange, men har samlet set været tilfredsstillende.  