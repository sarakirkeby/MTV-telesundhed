\chapter{Organisation}

\section{Indledning}
I dette kapitel vil der analyseres på effekterne af indført telesundhed i Favrskov Kommune som en organisation. Der vil tages udgangspunkt i arbejdsgangen på et specifikt sundhedscenter i kommunen, Hadsten Sundhedscenter. 

Analysen vil primært fokusere på pilotprojektet udført af Favrskov kommune, da projektet ikke er blevet integreret til fulde endnu. Der er på baggrund af samtaler med kommunen udviklet en case, som beskriver en typisk situation, hvor der bruges et virtuelt opkald i stedet for et fysisk møde. Det er ud fra denne case, at dele af analysen vil tage udgangspunkt.

Afsnittet har følgende fokusspørgsmål:
\begin{itemize}
\item{Hvilke organisatoriske betydninger er der ved implementering og drift af virtuel hjemmepleje med videokonference sammenlignet med konventionel fysisk hjemmepleje i Favrskov Kommune?}
\end{itemize}
Spørgsmålet søges besvaret med udgangspunkt i følgende punkter:
\begin{itemize}
\item{Forskel i medarbejdernes arbejdsgange før/efter virtuel hjemmepleje}
\item{Medarbejdernes reaktion}
\item{Beslutningsgrundlag for valg af Appinux-løsningen Det her har vi altså ikke en dyt om. Lad os erstatte med Implementation.}
\end{itemize}

\section{Case}
Casen er blevet udviklet i samarbejde med Favrskovkommune [Bilag 4, 4.1], og beskriver hvordan en typisk medicingivningssituation vil foregå, med og uden Appinux-løsningen implementeret. 

% Please add the following required packages to your document preamble:
% \usepackage{graphicx}
\begin{table}[]
\centering
\caption{Case for video opkald i Favrskov kommune}
\label{my-label}
\resizebox{\textwidth}{!}{%
\begin{tabular}{l|ll}
\textit{\textbf{}}    & \textit{\textbf{Manuelt besøg}}                                                                                                                                                                                                                                & \textit{\textbf{Virtuelt besøg}}                                                                                                                                                                                                                \\ \hline
\textit{Forberedelse} & \multicolumn{1}{l|}{\begin{tabular}[c]{@{}l@{}}Der aftales mellem borger og sundhedsprofessionelle hvordan\\ mødet skal foregå.\\ \\ Borgeren bliver skrevet på kørelisten i forhold til aftale.\\ \\ Medicinen er opdelt i ugebokse på forhånd.\end{tabular}} & \begin{tabular}[c]{@{}l@{}}Der aftales mellem borger og sundhedsprofessionelle hvornår\\ mødet skal være.\\ \\ Borgeren bliver skrevet på den separate køreliste til\\ videoopkald.\\ \\ Medicinen er opdelt i ugebokse på forhånd\end{tabular} \\
\textit{Under}        & \multicolumn{1}{l|}{\begin{tabular}[c]{@{}l@{}}Den sundhedsprofessionelle kører ud til borger.\\ \\ Den sundhedsprofessionelle hjælper borgeren med at tage sin\\ medicin.\\ \\ Den sundhedsprofessionelle forlader borger.\end{tabular}}                      & \begin{tabular}[c]{@{}l@{}}Den sundhedsprofessionelle ringer op.\\ \\ Borger tager røret og bliver guidet igennem\\ medicintagningen.\\ \\ Den sundhedsprofessionelle vurderer opgaven som gjort og\\ afslutter opkaldet.\end{tabular}          \\
\textit{Efter}        & \multicolumn{1}{l|}{Opgaven bliver vinget af som gennemført}                                                                                                                                                                                                   & Opgaven bliver vinget af som gennemført                                                                                                                                                                                                         \\
\textit{Hændelser}    & \multicolumn{1}{l|}{Op til 5 gange i døgnet}                                                                                                                                                                                                                   & 2-3 gange i døgnet                                                                                                                                                                                                                              \\
\textit{Tid}          & \multicolumn{1}{l|}{10 minuter}                                                                                                                                                                                                                                & 2-3 minutter                                                                                                                                                                                                                                   
\end{tabular}%
}
\end{table}


\section{Metode}
Informationer i dette afsnit baseret på data indsamlet ved hjælp af møder og e-mail-korrespondancer med repræsentanter fra Favrskov kommune. Disse informationer er understøttet af data fra en litteratursøgning i videnskabelige databaser. Litteraturstudiet er beskrevet detaljeret i Metode-afsnittet. 

Diskussionen vil tage udgangspunkt i artiklen ”Organisatiorisk implementering af informations- og kommunikationsteknologi”. Denne artikel deler implementeringen af en ny teknologi op i tre dele. De tre paragraffer af diskussionen vil repræsentere projektet i Favrskov Kommune, i forhold til disse aspekter. 

\section{Resultater}
\subsection{Ændringer i arbejdsgange}
Favrskov kommune deler overordnet ældreområdet op i tre; plejecentre, hjemmehjælp og visitationen. Plejecentrene er yderligere delt i syd, nord og vest, og hjemmehjælpen er delt i øst og vest. Hadsten sundhedshus hører under hjemmeplejen og ligger under organisationens østlige afdeling. Hele kommunen er opbygget op efter BUM-modellen. Video opkald er forsøgt implementeret i de ydreområder. Projektet er forsøgt implementeres med henblik på, at der skal spares minutter i sygeplejernes arbejdsdag. Dette skal give plads til andre opgaver, og derved mulighed for økonomiske besparelser i kommunen. 

De sundhedsprofessionelle i hjemmeplejen er overodnet delt i fire teams, som svarer til to i hvert distrikt. Der er arbejdes i teams med op til 25 medarbejdere i hver. Ud af disse er 3 i hver gruppe ansvarlig for videoopkald på nuværende øjeblik, men der er flere som udføre dem.  

Appinux’s løsning er tiltænkt som en erstatning for de medicingivningsbesøg som kommunen ellers tilbyder. Medicingivning er et tilbud, som hjælper patienter med at indtage deres medicin. Denne opgave afhænger af at der foregående er sket medicinadministration. Medicinadministation skal foretages af en sygeplejerske frem for en hjemmeplejer, og består i at dele medicinen op i korrekte portioner til hver dag i ugen.

Før Appinux’s løsning blev implementeret, kørte de sundhedsprofessionelle ud til borgeren, hver gang borgeren skulle indtage medicin [Bilag 4, 4.1]. Med Appinux løsning, sidder de sundhedsprofessionelle foran en tablet og ringer borgeren op. Borgeren er ved hjælp af en tablet i deres eget hjem, i stand til at høre og se den sundhedsprofessionelle, og kan derved blive guidet igennem medicingivningen. Mere information om Appinux-løsningen kan hentes fra det tekniske afsnit. 

Der er etableret et call-center i Hadstens sundhedscenter, hvor de sundhedsprofessionelle kan foretage opringingerne til borgerne[Bilag 11, 11.28]. Videoopkaldene kan enten foretages i dette call-center, eller det kan foretages på en transportabel tablet. Call-centeret var etableret før video-opkald kom på tale i kommunen, og blev brugt som en regulært call center, hvor borgere kan ringe ind, hvis de har brug of hjælp.

De sundhedsprofessionelle holder øje med observationsoverblikket i Appinux, og kan her se hvem der skal ringes til, og om en borger har ringet til centralen uden at have en aftale med centeret. Der er op til tre sundhedsprofessionelle som har ansvaret for video opkald fra callcenteret, og der aftales internt i denne gruppe, hvem der er designeret de forskellige opkald.  Der aftales internt i blandt de sundhedsprofessionelle om morgenen, hvem der bliver designeret de forskellige opkald.

Appinuxsystemet blev oprindeligt implementeret med modulet TOBS, men Appinux har senere givet Favrskov kommune lov til at afprøve video opkald i kommunen. I forbindelse med at systemet blev implementeret, blev der oprettet superbrugerroller, som blev pådraget enkelte sundhedsprofessionelle. Disse superbrugere har det overordnede ansvar omkring applikationen. De sørger for oprette nye brugere i systemet, samt at slette brugere i tilfælde af for eksempel død. Superbrugerne holder styr også styr på at opdatere ændringer i sammenhæng med borgere, såsom adresse ændring og reorganisering af teamsne.

Der er lavet vejledninger til superbrugerne, som detaljeret beskriver fremgangsmetoden til de forskellige scenarier [Bilag 13, 13.1]. Superbrugerne holder møde en gang hvert halve år. Ud over disse superbrugere har Karin Juhl og Rekha Kotyza hovedansvarlige for projektet. 

\subsection{Implementering}
Virtuel hjemmepleje blev først afprøvet i Favrskovkommune i form af et pilotprojekt, som blev udført i starten af 2015. Det primære ansvar for implementering af teknologien, har lagt ved Karin Juhl og Rekha Kotyza. Implementeringen har således ikke været drevet af Appinux, som dog har givet indledende support om blandt andet valg af udstyr. Pilotprojektet var centraliseret omkring Hadstens sundhedscenter, og gjorde brug af borgere over i hele Favrskov kommune. 

Kommunen forventer at projektet er færdigimplementeret i de to ydre distrikter af Favrkov kommune i maj 2016. I forbindelse den fulde implementering, forventes det at videoopkald kommer til at blive brugt af samtlige medlemmer af de fire teams. Kommunen forventer at det er klart til, at visitationen kan overtage projektet fra oktober, og derved kan tilbyde video opkald i stedet for fysiske forsøg, til borgere som er egnede.

Karin Juhl og Rekha Kotyza har sørget for give undervisningstimer til de sundhedsprofessionelle som har været berørt af pilotprojektet. Desuden blev der ved opstart af projektet, givet vejledninger til de sundhedsprofessionelle, som beskriver opkalds forløbet til borgerne [Bilag 4, 4.1]. Denne vejledninger er udstyret med billeder for lettere forståelse. Vejledningen dikterer samtidig opførsel i skærmopkaldet med henblik på at der skal være så minimal en forskel mellem et fysisk besøg og et skærmbesøg, og samtidig give tryghed til brugeren.

Denne vejledning har vist sig brugbar for de sundhedsprofessionelle. De vil dog gerne have en ajourført guide i fremtiden, og efterspørger generelt undervisning i applikationen.

 
\subsubsection{Modtagelse og støtte}
Projektet blev først introduceret til de berørte teams, som en anderledes måde at udføre en opgave, som allerede blev tilbudt af teamene. Det blev introduceret som en obligatorisk opgave, og blev præsenteret af teamlederne. Førstehåndsmodtagelsen var blandet. 

Favrskov kommune lavede en interview undersøgelse, hvor to medarbejdere fra pilotprojektet svarede på spørgsmål. Disse fortæller at medarbejdere som var indblandede, også havde blandede følelser omkring applikationen. Interviewundersøgelsen fortæller også at medarbejderne har reflekteret over hvordan videoopkald kunne videreudvikles i kommunen. 

Der har været positiv respons fra mange af de sundhedsprofessionelle omkring projektet. Responsen påpeger primært hvor meget tid det er muligt at spare.

De primære negative følelser omkring projektet omhandler de tekniske problemer som de sundhedsprofessionelle stødte ind i sammenhæng med applikationen. Specifikt oplevede de sundhedsprofessionelle problemer med forsinkelser med lyd og forringet billedkvalitet. De er desuden opmærksom på at der skal være strenge regler omkring hvilke borgere der skal godkendes til at være egnede til videoopkaldsydelsen. Dette involvere folk som er kognitiv dårlige, eller ældre som har dårlige tekniske evner. 

Efter pilotprojektets opstart har der været et møde omkring projektet, hvor involverede sundhedsprofessionelle og projektansvarlige har evalueret på projektets udførelse. På dette møde blev der nævnt de samme tekniske problemer, som der tideligere var hørt fra medarbejdere. Der blev desuden fastlagt procedurer, for scenarier der kunne udfoldes i forbindelse med ikke at kunne få kontakt med borgeren. 

Det har til dagsdato ikke været muligt at finde informationer vedrørende beslutningstagningen for at indføre projektet.

  
\section{Diskussion}
\subsubsection{Organisationens forudsætninger}
Appinux’s løsning blev ikke introduceret til Favrskov kommune som en løsning på et problem. Dette er i uoverensstemmelse med organisationsimplementeringsmodellen fra Ikt og læring – reflekteret praksis. I stedet blev projektet introduceret som en ændring i arbejdsteknik. Der er ikke blevet ført en indledende undersøgelse på traditionelvis, da Ældrechef Peter Mikkelsen, er blevet bekendt med projektet igennem Halsnæs kommune. Appinux’s løsning blev oprindeligt indført med modulet TOBS, og er senere blevet udvidet med modulet Video Opkald. Videoopkald blev introduceret som en obligatorisk opgave, med tanke på at erstatte nogle af de mindre opgaver som hjemmehjælperne står over for i deres hverdag. 

Projektet er forsøgt præsenteret på en måde så det så vidt som muligt ligner medarbejdernes tideligere arbejdsgang. Der er udvidet i de korrekte organisatoriske afdelinger, så som udvikling af superbrugere, definition af ansvarsområder og fastlæggelser af scenarier. 

\subsubsection{Medinddragelse af brugere}
I introduktionsfasen blev projektet modtaget med blandede følelser. Dette kan underbygges af mangel på introduktion hos medarbejderne, som ”Ikt og læring – reflekteret praksis”’s organisationsmodel foreslår. 

Igennem implementeringsfasen har medarbejdernes, trods mærkbare tekniske problemer, tanker omkring virtuelle besøg primært været positive. De sundhedsprofessionelle mener at deres arbejder bliver nemmere ved hjælp af denne teknik. Der har altså ikke været nogen specifikke frygt for at arbejdet kan erstatte arbejdsopgaver som de sundhedsprofessionelle på nuværende tidspunkt udfører, som der f.eks. har været grundlag for i et randomniseret forsøg i det nordlige England\cite{Mair}. Tværtimod har støtten været overvældende, og flere af de sundhedsprofessionelle kan se mulighederne i projektet og selv kommet med forslag til at udvide applikationen. Det virker som om organisationen kollektivt er vokset til at se applikationen som et hjælpemiddel, frem for en erstatning.

Det er tydeligt at mærke at der er enthusiasme omkring projektet fra de sundhedsprofessionelle. Det er endeløst så mange muligheder de kan se i projektet, og det er tydeligt at projektet har været godt støttet op omkring fra ledelsens side. Dette giver også projektet bedre muligheder fra start, da der er større risici for at et projekt mislykkedes, hvis der er ikke støtte omkring projektet fra lederne\cite{Ikt}. 

\subsubsection{Differentieret uddannelse af brugere}

For at et system kan fungere optimalt, kræver det at brugerne har forudsætninger for at bruge systemet optimalt. Dette kræver undervisning\cite{Ikt}. I Favrskov kommune har der været to designerede personer som sørger for undervisning i Appinux’s løsning. 

Arbejdsgangene i Favrskov kommune har ændret sig marginalt i forhold til hvordan det fungerede før i tiden. Ændringen der er foretaget er i stedet en ændring i arbejdsmetode, idet de sundhedsprofessionelle sidder foran en tablet, og udfører det arbejde, som de ellers ville have kørt ud til borgeren for at udføre. Dette passer også overens med, at videopkald i visse kredse, ikke sanses som en ændring i en arbejdsgang, men istedet bliver defineret som en ny arbejdsteknik\cite{telenursing}.  

Et problem som de sundhedsprofessionelle hurtigt tog til sig, var problemerne med at borgere måske ikke var kompatible med den ydelse der kunne tilbydes nu. En bekymring var at de ældre, de tager sig af, måske ikke har de mentale eller tekniske egenskaber, der skal til, for at kunne udføre et videomøde. Dette er dog imødekommet ved at starte ud småt, og kun tilbyde støtte til medicintagnings- og måltidshjælp. Problemer kan dog stadig forekomme i takt med at ydelsen bliver udvidet, og flere kan blive egnet. Dette kan for eksempel være, at nogle ældre som er egnet til at modtage ydelsen, har deres medicin låst væk. På nuværende tidspunkt har Favrskov kommune ingen måde at kunne låse en medicin op digitalt, og derved vil videoopkald ikke være egnet i den sammenhæng [Bilag 4, 4.1]. Desuden vil visitationen altid foretrække at give ældre rehabiliteringskurser, frem for at tilbyde ydelser. Med medicingivning giver du ydelser, og ikke selvoptræning. 


\section{Delkonklusion}
De sundhedsprofessionelle har reageret med blandede følelser omkring projektet. Overordnet var modtagelsen positiv blandt de sundhedsprofessionelle, på trods af mærkbare tekniske problemer. Medarbejderne kan desuden se fordele ved at udvide videoopkald til andre aspekter af deres arbejdsdag. 

Der er kun sket små forskelle i arbejdsgange i kommunen i forbindelse med reelt arbejde. Organisatorisk er der sket en større forskel, idet der er etableret en ny arbejdsgruppe, samt nye nøgle figurer.
