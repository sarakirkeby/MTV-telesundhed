\chapter{Organisation}

\section{Indledning}
I dette kapitel vil der analyseres på effekterne af indført telesundhed i Favrskov Kommune som en organisation. Der vil tages udgangspunkt i arbejdsgangen på et specifikt sundhedscenter i kommunen, Hadsten Sundhedscenter. Der vil sammenlignes arbejdsgange for før, og efter, denne metode blev introduceret til arbejdspladen. Analysen vil primært fokusere på pilotprojektet udført af Favrskov kommune, da projektet ikke er blevet integreret som en del af den almene arbejdsgang endnu. 

I dette kapitel vil følgende spørgsmål forsøgt besvaret:
\begin{itemize}
	\item Hvilke forudsætninger er der, for at video-tele-conferencing i Favrskov kommune fungerer optimalt? 
	\item Hvordan er reaktionerne fra de sundhedsprofessionelle? 
\end{itemize}

\section{Metode}

\section{Resultater}
\subsection{Ændringer i arbejdsgange}
Projektet er forsøgt implementeres med henblik på, at der skal spares minutter i sygeplejernes arbejdsdag. Dette skal give plads til andre opgaver, og derved mulighed for økonomiske besparelser i kommunen. 

Der er med andre ord, lavet en ændring af arbejdsteknik blandt plejerne\footnote{J Telemed Telecare-2001-Arnaert-311-6}.

Videoopkald kan enten foretages fra call-centeret i Hadsten Sundhedscenter, eller det kan foretages på en tablet, som de sundhedsprofessionelle har med sig. Den sidstnævnte løsning, er dog ikke implementeret endnu.

Videoopkald er forsøgt præsenteret på en måde så det så vidt som muligt ligner medarbejdernes tideligere arbejdsgang. Der er lavet separate køreliste til videoopkald.

\subsection{Implementering}
Virtuel hjemmepleje blev først afprøvet i Favrskovkommune i form af et pilotprojekt, som blev udført i starten af 2015. Det primære ansvar for implementering af teknologien, har lagt ved Karin Juhl og Rekha Kotyza. Implementeringen har således ikke været drevet af Appinux, som dog har givet indledende support om blandt andet valg af udstyr. Pilotprojektet var centraliseret omkring Hadstens sundhedscenter, og gjorde brug af borgere over i hele Favrskov kommune.

Pilotprojektet forventes færdigimplementeret i de fire distrikter af Favrkov kommune i maj 2016. Kommunen forventer at det er klart til, at visitationen kan overtage projektet fra oktober samme år, og derved kan tilbyde video opkald i stedet for fysiske forsøg, til borgere som er egnede. 

Til gruppens viden, blev der ikke givet undervisningstimer i Appinux applikationen til de sundhedsprofessionelle. I stedet blev der ved opstart af projektet, givet vejledninger til de sundhedsprofessionelle, som beskriver opkalds forløbet til borgerne\footnote{Videoopkald til borger i Appinux}. Denne vejledning er udstyret med billeder for lettere forståelse. Vejledningen dikterer samtidig opførsel i skærmopkaldet med henblik på at der skal være så minimal en forskel mellem et fysisk besøg og et skærmbesøg, og samtidig give tryghed til brugeren. 

Denne vejledning har vist sig brugbar for de sundhedsprofessionelle. De vil dog gerne have en ajourført guide i fremtiden, og efterspørger generelt undervisning i applikationen.
 
\subsection{Modtagelse og støtte}
Hvis ledelsen i en organisation, ikke støtter op om en løsning der skal implementeres, så går implementeringsprocessen sværere, end hvis det ikke havde været tilfældet. I Favrskov kommune har der været støtte omkring projektet\footnote{???}. 

Der har dog været mærkbare tekniske problemer, så som forsinkelser i lyd eller billedkvalitet. Der er også betydeligt forskel 

Det er bevist, at hvis ledelsen, eller medarbejderne, er modvillige til at implementere en teknologi, så er der større risici for at det mislykkedes. 

\subsection{Intern evaluering}
Der er blevet udført intern evaluering i Favrskov kommune, efterfølgende pilotprojektet. 

Entusiasmen for projektet, kan yderligere ses ved at medarbejderne giver forslag til hvad projektet kan bruges til i fremtiden.

I den interne evaluering, er der udarbejdet retningslinjer for, hvad der skal ske i forskellige scenarier, hvor borger ikke svarer på deres tablet. 

Der er dog også kritik iblandt de sundhedsprofessionelle, da de også se hvem der eventuelt ikke skal være en del af projektet, som f.eks. kognitivt dårlige ældre eller ældre med dårlige tekniske evner. 

Dette projekt er testet, men er ikke blevet implementeret helt endnu, da det på nuværende dato, kun er udvalgte sundhedsprofessionelle som bruger løsningen. 


  
\section{Diskussion}

\section{Konklusion}
Overordnet var modtagelsen positiv blandt de sundhedsprofessionelle, på trods af mærkbare tekniske problemer.

Medarbejderne kan desuden se fordele ved at udvide videoopkald til andre aspekter af deres arbejdsdag.  
