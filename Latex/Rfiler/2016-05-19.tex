\chapter{Mødereferat}

\section{Dato: 19.05.16}
\hrule

\textbf{Fremmødte:} Lise, Melissa, Jakob, Jeppe, Sara og Mohamed \\
\textbf{Fraværende:} Ingen
\\textbf{Vejledere:} Jesper
\\
\\
\textbf{Dagsorden}
\begin{itemize}
	\item Status
	\item Tjek af Fokuserede Spørgsmål
	\item Bilag
	\item Versionshistorik
	\item Gyldigheden af Struktureret Interviewundersøgelse fra Hadsten
	\item Reference system
	\item Diverse spørgsmål 
\end{itemize}


\textbf{Referat} 
\\
\textbf{Status:}\\
Det er fedt at vi er kommet ordentligt på vej! Der kan indrages i diskussionen at vi har en tilsvarnde case fra viborg. Der kan sammenlignes imellem de to.\\ 
Teknologiafsnitet er blevet rettet i forhold til feedback fra Jesper. Vær lidt mere konkret.\\ 
\textbf{Fokuserede spørgsmål:}\\
Der er lavet ny fokuserede spørgsmål - OK! Skarpeheden skal være fin. Vi er nået til noget som vi kan svare på i de her spørgsmål. De er fine og godkendte. \\
Teknologiafsnittet skal komme ind på noget teknisk implementering.\\ 
\textbf{Bilag:}\\
Mails er bilag. Det er vigtigt at kunne vise at vi har stillet nogen ordentlige spørgsmål og hvilke svar vi har fået fra dem. Hvis der er nogen vigtigt kan vi eventuelt citere fra bilaget. Der refereres til bilag - også mails.\\ 
\textbf{Versionshistorik:}\\
Der skal ikke være versionshistorik. Med hensyn til mails, skal de være med. \\
\textbf{Gyldigheden af interviewundersøgelse:}\\
Interview undersøgelserne kan være med, men vi skal være kritiske. Man har indsamlet erfaringer systematisk, så de kan godt bruges. Bare husk at være kritisk!\\ 
\textbf{Referencesystem:}\\
Er blevet enige om at vi skal bruge Vancouver. Bente stoler på at det som Rasmus har vist passer. Det skal bare være rigtigt!! Projektstyringen skal stå nede i bilag, hvordan der er arbejdet, og hvem der har styret og så videre. \\
Bente vil gerne have en URL på på hjemmesider. Alle artikler skal ikke ligge som bilag da de er direkte kilder. Vi dobbeltjekker alt reference igennem for at sikre at det står rigtigt. \\
Vi skal henvise i forhold til hvilken model vi bruger. Ikke hvad de definerer i stoffet. Vi skal stole på systmet.\\

\textbf{Diverse Spørgsmål:}\\
\textit{Må man skrive, at et afsnit bygger på en kilde og så ikke skrive kildehenvisninger?}\\
Ja det må man godt!\\

\textit{Er indledningen stadig for bred?}\\
Den er alt alt for bred. Det handler ikke om ændrebyrden på nogen måde. Det skal være konkret i forhold til favrskov kommune. Hele indledning skal rettes ind så det passer direkte tl den case som vi arbejder med. Alt det overordnede med Danmark skal vi ikke arbejde med her. Bare fokuser på casen. \\
Hvad skal figurerne bruges til i den her indledning? De kan kun bruges til at vise at udviklingen ligger pålandet og ikke i storbyerne. Billederne skal relateres i forhold til vores specifikke case. Vi behøver ikke figurerne. De bliver lidt fyld her. \\ 
Det kan koges ned til få linjer, og så derefter skal der udybes i forhold til Favrskov i stedet.\\ 
\\
\textit{Hvor skal de fokuserede spørgsmål stå? }\\
Der skal være fokus over det hele. Jesper: De skal stå på indledningen og så er det det man svarer på.\\
Det er OK at de står over de forskellige afsnit. Det gør ikke noget. Det skal bare stå i indledningen.\\
\textit{Hvor omfattede skal interessentanalysen være?}\\
Bente kan godt finde på at stille spørgsmål til interessant analysen til eksamen. Det er ligemeget for bente om det står i rapporten eller ej.\\
\textit{Skal kildehenvisninger være før eller efter punktum?}\\
Det er måske defineret af reference systemet. Jesper og Bente siger at det skal være før. DET SKAL VÆRE FØR.\\ 
\\
\textit{Skal der være kilder på i diskussionsafsnittet – hvis f.eks. jeg skriver ”på baggrund af afsnittet ”resultater” kan det diskuteres…”?}\\
Det kommer an på om man bruger det samme materiale som man brugte oppe i afsnittet. Bente og Jesper vil gerne have at der er!!\\

