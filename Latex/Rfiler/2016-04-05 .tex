\chapter{Mødereferat}

\section{Dato: 05.04.16}
\hrule
\textbf{Fremmødte:} Lise, Melissa, Jakob, Jeppe, Sara og Mohamed \\

\textbf{Fraværende:} Ingen
\\
\\
\textbf{Dagsorden}
\begin{itemize}
\item[Spørgsmål/Forberedelse til Appinux mødet]
\item[Gennemgang af Link på BB]
\end{itemize}

\textbf{Referat} 
\\
Der ligger et link inde på BB som beskriver hvilke kriterier der skal til, for at man kan se det som telemedicin. Skim dette igennem, sådan vi kan få en ide om hvad det er der forgår, og så vi kan stille intelligente spørgsmål.\\
\\
Strategiske samarbejdspartnere kan f.eks. være Falck. Hvordan kommunikerer de med dette system? Har der være behov for mere?\\
\textbf{Appinux}
\begin{itemize}
\item[Sikkerhed for dataoverførsel]
\end{itemize}
Det er kun dem der skal se dataen, som har set dem. Det er følsomme oplysninger, og der må skulle være nogen minimumskrav. Her kan der spørges appinux, om hvilke sikkerhedsstandarder som de opfylder. De billeder der bliver sendt frem og tilbage er personlige, og vi er interesseret i at der ikke bliver lækket noget. Læs eventuelt det link som Jesper har lagt op. Det gør ikke nogen forskel i forhold til sikkerhed hvilken browser det kører igennem. Man kan dog godt være udsat for at patienterne bliver overtalt til at installere skadeligt software, som kan lave en fjendtlig overtagelse af enheden. Eks. Google Cast, Team Viewer osv. Vi skal ikke bruge meget tid på det, vi skal bare vide at det er der. \\
\begin{itemize}
\item[Minimumskrav til internetforbindelse]
\item[Krav til computer/Tablet]
\end{itemize}
Der er begrænsninger på hvad der kan køre på en bestemt internet forbindelse. Disse begrænsninger vil fortsat blive der, medmindre at kommunen f.eks. laver en aftale med kommunen. \\
Der må være nogen minimumskrav til hardwaren. Hvor ligger ansvaret hvis noget går galt? Hvem skal større for vedligeholdelsen af udstyret?\\
Jesper foreslår at man tager udgangspunkt i sine egne erfaringer med internettet og eventuelt finder hjælp med det link der er givet, eller hos ham.\\ 
\begin{itemize}
\item[Open source/connections]
\item[Standarder - Continua Health Alliance]
\end{itemize}
Hvad skal vi være opmærksomme på? Er der mulighed for at vidreudvikle på det her? Kan det optimeres til f.eks. at give sårsupport? Disse standarder kan findes ved Continua Health Alliance. Standarderne beskriver hvilke krav der er til udstyr. Hvis de bruger open connections, så er forbindelserne kompitabel, og så skal der altså ikke laves om i forbindelserne for at det kan snakke sammen. Kan Appinux løsningen snakke sammen med f.eks. en blodryksmåler? En ting er at vi har apperater som kommunkerer. Et andet er systmet.
\\
Open source - Koden der driver værket. 
Open connections - Kan snakke sammen med andre ting som fungerer ved hjælp af en eller anden form for standard.\\
\begin{itemize}
\item[Omkostninger af vedligeholdigelse og Opdatering af systemet og server]
\end{itemize}
Angående problemet omkring test, så lyder det mærkeligt, at Favrskov kommune selv skulle teste tingene. \\
\begin{itemize}
\item[App Vs Chrome-Browser]
\end{itemize}
Web connect i Chrome browser. Dette er et godt standard værktøj, og det er derfor at det bliver kørt i en Chrome browser i stedet for at køre i en App. Dette gør også at det kører på mange forskellige tablets. De sagde at det kørte i en browser. 
\begin{itemize}
\item[Codec - Hvordan sikrer vi kvalitet?]
\end{itemize}

\textbf{Konkrete spørgsmål}
\begin{itemize}
\item[Hvilke sikkerhedsmekanismer opfylder i?]\\
\item[Hvilke overvejelser har i gjort for at imødekomme sikkerhedsbrud?]\\
\item[Hvilke minimumskrav har i til internetforbindelsen?]\\
\item[Hvad er jeres forudsætninger, for at jeres udstyr skal køre optimalt? (Internet/Hardware)]\\
\item[Hvem er ansvarlig for at de foregående forudsætninger er opfyldt?]\\
\item[Hvilke standarder opfylder i? (Eks, Continua Health Alliance)]\\
\item[Kan platformen udvides til at kommunikere med andre systemer? (Open Connections)]\\
\item[Er platformen open source?]\\
\item[Hvad er kotymen når der kommer en ny version?]\\
\item[Hvad menes der konkret, når der siges at Favrskovs kommune selv skal teste nye funktioner?]\\
\item[Er det en app som kører i en browser, eller kører browseren i en app?]\\
\item[Hvem driver/er ansvarlig for serveren og hvor står de henne?]\\
\item[Hvem er jeres strategiske samarbejdspartnere/fagsystemsleverandører og hvad menes der med dette?]\\
\item[Codec]
\end{itemize}