

\section{Dato: 12.04.2016 - Møde med Appinux}
\hrule

\textbf{Fremmødte:} Lise, Sara, Mohammed, Jeppe, Jakob, Mette fra NetPlan og Michael fra Appinux \\
\textbf{Fraværende:} Melissa 
\\
\\
\textbf{Dagsorden}
\begin{itemize}
	\item Generel information
	\item Favrskov Kommune
	\item Appinux' løsning - virtuel hjemmepleje
	\item Minimumskrav
	\item Standarder
	\item Support
	\item Sikkerhed
	\item Internetforbindelse
	\item Videreudvikling
	\item Implementering
	\item Økonomi
	\item Generelt om telemedicin
\end{itemize}

\textbf{Generel information} 
\\
Navn: Michael Ellegård \\
Firma: Appinux \\
Titel: Ejer, sammen med 2 andre

Appinux er et firma der udvikler vha. crowd sourcing. Der er derfor en masse forskellige mennesker rundt omkring i verden der sidder og udvikler på deres applikation. Personer som de ikke møder real life, da de i princippet ikke er "ansat".

De arbejder indenfor følgende 3 punkter:
\begin{itemize}
	\item Jobcenter 
	\item Social psykiatri
	\item Sundhedsområdet
\end{itemize}
Det er samme platform som bruges til de 3 forskellige områder.\\


\textbf{Favrskov Kommune} 
\\
Favrskov købte tidlig opsporing af Appinux og ikke video!\\
Men de har fået lov at bruge det.

I Favrskov Kommune benyttes der tidlig opsporing, støtteværktøjer og skærmbesøg.

Statistik for Favrskov Kommune: \\
431 medarbejdere \\
1375 borgere\\
703 videomøder

Login: \\
Jan = 978 \\
Feb = 1134 \\
Mar = 906

Videoopkald: \\
Jan = 102 \\
Feb = 137 \\
Mar = 46

Kalenderaftaler: \\
Jan = 1064 \\
Feb = 1191 \\
Nar = 1505

Fokus områder:\\
Man vælger måske kun mellem 70-80 mennesker og spørger om de vil have Video Tele Conferencing eller varme hænder, og så kører man fuld skrue på at implementere det der. Det gav mange flere videomøder tiltrods for at det var færre borgere.


********** Triasering?? Noget med nogle krav til patienten og hvilken tilstand denne er i. ***********


\textbf{Appinux' løsning - virtuel hjemmepleje} 
\begin{itemize}
	\item Skal være en chrome browser.
	\item De har også en APP, som så åbner en chromebrowser.
	\item De logger tidspunkt og varighed af opkaldet.
	\item Video og tale hænger sammen. De er altid synkroniseret og hvis en af dem ikke virker, så stopper den anden også med at virke.
	\item Den platform Appinux har kan bruges til mange ting. Tidlig opsporing, skærmbesøg, telemedicin(direkte måling af kol, hjerte mv.), teamtavler, telesundhed.
	\item Køberen kan selv bestemme hvilke dele der skal være på platformen. 
Der er mere end 70 moduler i Appinux platformen, som man kan vælge imellem.
Det kan styres hvilke moduler den enkelte skal have adgang til, således at alle ikke har adgang til de samme data.
\end{itemize}


\textbf{Minimumskrav}
\begin{itemize}
	\item Hardware/software: De ved ikke hvad minimumskravet er.
	\item De har nogle anbefalinger - fx ikke under android 4.02.
	\item Ikke et reelt krav til app og browser, men der skal 512 kbit/sek til for at der kan køre video.
\end{itemize}


\textbf{Standarder}
\begin{itemize}
	\item De bruger gerne standarder hos Appinux!
	\item WebRTC understøttet af google hangout(standard) som de bruger. 
	\item Video standard(google hangout) lavede til almindelige mennesker.
	\item Apparater = continua health alliance
	\item Video = web RTC
	\item Sundhedsbaserede integrationer = HL7 - herunder FHIR
	\item Personjournal = PHMR (personal health monitoring record)
	\item Lagring, dansk ift. telemedicinske data = XDS-lagring. \\IHE profiler(navn, sted, tid, måling mv.) der kan gemmes i XDS. Det er en international standard som DK gerne vil rulle på nationalt plan.
	\item Kommunerne har en særlig selvlavet standard - fællessprog 1 og 2. De er ved at udvikle 3'eren. \\ Den ligger Appinux sig op af. 
\end{itemize}


\textbf{Support} 
\\
Michael: "Support er et kæmpe problem!" \\


\textbf{Sikkerhed}
\begin{itemize}
	\item Al trafik er krypteret (https).
	\item Alle data er krypteret.
	\item Password der skal være en vis længde. Der er dog mulighed for at sætte applikationen til at huske ens password, hvilket giver potentiale for sikkerhedsbrud.Der kunne laves noget fingeraflæsning eller lignende. 
	\item Der gemmes først lokalt på medarbejderens telefon og så sendes data når der er hul igennem.
	\item Det samme gælder ikke for borgeren, da man ikke må gemme på borgerens telefon.
	\item Appinux har valgt ikke at gøre det, så borgeren skal være online for at det kan gennemføres. Det medfører at Appinux er sikre på at borgeren tager kontakt til sygehuset hvis han/hun er i fare. Det er en grundlæggende sikkerhedsforanstaltning.
	\item Hvis patienten ikke svarer sendes der en hjemmeplejer ud til dem.
	\item Alt på medarbejdersiden er logget, så man kan se hvad de laver.
	\item Der er nogle superbrugere der kan kontrollere, om folk holder sig inde for deres beføjelser. 
	\item Man kan som borger se hvem der har kigget på dine personlige data, og klage hvis en ukendt læge, sygeplejerske el. har været inde og kigge ens profil.
	\item Pårørende kan få adgang til data, hvis borgeren giver adgang til det. Samtykke fra borgeren styrer hvem der har adgang til hvilke data.
\end{itemize}


\textbf{Internetforbindelse}
\begin{itemize}
	\item Benytter virtuel bitrate.
	\item Kører på alle bredbåndsforbindelser.
	\item Ligger et simkort i en tablet, giver det til borgeren og så skal det fungere.
\end{itemize}

\textbf{Videreudvikling}
\begin{itemize}
	\item Alle kan i princippet lave apps til appinux' system, men de skal have en rolle og adgang til de forskellige ting. Databehandleraftale.
\end{itemize}


\textbf{Implementering} 
\\
Lovgivning og kultur spiller en stor rolle når man skal implementere it indenfor sundhed.\\
Appinux vil ikke implementere! \\
De leverer en startpakke, men de vil ikke hjælpe med selve implementeringen. 
Hans erfaring er, at kommunerne er ikke gode til at implementere, så det kunne være en idé at skaffe en implementerings partner.
Kommunerne tror at de kan klare det selv, men det kan de ofte ikke.
Et eksempel er, at Favrskov Kommune kørte på tablets der var 3 år gamle og så fungerede Appinux' system ikke.\\


\textbf{Økonomi} 
\\
SKI aftale.

Appinux leverer denne pakke:\\
Abonnement
\begin{itemize}
	\item Drift. \\ Besværligt med apps. Kommunen tager selv ansvar for systemet/produktet. Hvis kommunen opdaterer fx 40 tablets hvorefter de ikke fungerer, så er det kommunens problem. Appinux har alt virtuelt.
	\item Support: (redmine). \\ Ingen stor support. De kommer ikke fysisk ud. De har heller aldrig oplevet efterspørgsel på det. Problemet modtages af en dansker og sendes videre til en 'medarbejder' i udlandet i tilfælde af at problemet er meget teknisk.
	\item Vedligeholdelse.
	\item Videreudvikling: 4 releases om året. \\ Kommunerne har 2 systemer. Presystem, uddannelsessystem og Produktionssystem.\\
Der kommer en opdatering til presystemet og så tester de det, og kommer med input til hvordan det fungerer, og så laver Appinux eventuelle ændringer.\\
På den måde tester kommunen deres egen opsætning i sammenspil med den nye opdatering.\\
Appinux tester selv 3 uger før der begynder at komme kunder på. Det er tovejs kommunikation, når der skal laves en opdatering.\\
Favrskov skal sige til Appinux at de gerne vil opdatere, men de må ikke være mere end to opdateringer bagud.\\
Uddannelsessystemet bruges til eventuelle nye medarbejdere. 
Kommunerne kan selv vælge hvilken type abonnement de gerne vil have. Fx om de vil have en MDM pakke med, således at Appinux ????står for opdateringen?????.
\end{itemize}

Favrskov har fået en pakke til en vis pris.\\
En sosu medarbejder om måneden ca.!\\
2 indgangsfees(startpakken):\\
 - Instalation, 15.000\\
 - Implementering: opsætning, uddannelse, 57.500\\\\
Man betaler pr. måned for et abonnement: \\
Betaling foregår pr. antal aktive brugere. \\
50 brugers løsning med telemedicin(kalender, dataopsamling, video) =
10.000 kr. pr. måned. \\
Der er gratis benyttelse af fagkonsulenter.\\
Appinux har ansvaret for at serveren kører. Serveren kører i Skyen, men den står ved T26, som har adgang til sundhedsdatanettet.\\

2 prismodeller - tjek SKI 2.19 (cloud) og 2.07 med TDC
SE cloudfactory - Appinux ligger under dette firma:\\
- Grundmodel som er basis. 6500 kr. for 0-50 brugere\\
- Tidlig opsporing. 2000 kr.\\
- Telesundheds modulet. 3000 kr.\\\\
500 - 750\\
750 - 1200

Samarbejdspartnere:\\
Video og træning fx (exolife, aidq)


Appinux' kode er frit tilgængeligt, men man skal have et eller andet licens halløj???\\


\textbf{Generelt om telemedicin} 
\\
Tværsektorielt samarbejde er guld. IT er godt til at levere fakta.
Opdragende effekt ved at man kan komme direkte i kontakt med fx diabetes 'specialister', når man har en telemedicinsk løsning i og med at der sidder folk i et call center der har ekspertise inden for præcis dette område. 
Man slipper altså for at tage kontakt til sygehuset og blive sendt videre igen og igen.

Største udfordring:
Implementering fra en fysisk ydelse til en virtuel ydelse.
De ansatte kan evt. miste deres arbejde, og de skal samtidig omvendes fra deres uddanelse om fysisk kontakt til video.	

