

\section{Dato: 12.04.2016}
\hrule

\textbf{Fremmødte:} Lise, Sara, Mohammed, Jeppe og Jakob
\textbf{Fraværende:} Melissa 
\\
\\
\textbf{Dagsorden}
\begin{itemize}
	\item Appinux' løsning - virtuel hjemmepleje
	\item Anvendelse
	\item Formål
	\item Udbredelse
	\item Erfaring og udfordringer
	\item Dækningsproblemer
	\item Demonstration
	\item Andre løsninger/udviklingsplaner
	\item Økonomi
\end{itemize}

\textbf{Referat} 
\\
Michael Ellegård fra Appinux \\
Appinux har udviklet vha. crowd sourcing. Derfor en masse udlændinge der er udvikler, som de ikke møder real life. Freelance arbejde, men de ansætter folk for 6 måneder. 
3 ejere. De vil ikke rigtig have folk ansat, men mere at folk gør det af interesse og engagement. 
Lovgivning og kultur spiller en stor rolle når man skal implementere it indenfor sundhed.

Support:
Support er et kæmpe problem!\\
De kørte på tablets der var 3 år gamle og så kunne det ikke køre.\\
Appinux vil ikke implementere! \\
Leverer en pakke, men vil ikke hjælpe med implementeringen. 
Kommunerne er ikke gode til at implementere, så det kunne være en idé at skaffe en implementerings partner.
De tror at de kan det selv, men det kan de ofte ikke.\\
Triasering?? Noget med nogle krav til patienten og hvilken tilstand denne er i.
\\

\textbf{Appinux' løsning - virtuel hjemmepleje} 
\\
1. Jobcenter
2. Social psykiatri
3. Sundhedsområdet

Samme platform som bruges til forskellige områder.\\

Statistik for Favrskov Kommune:
431 medarbejdere
1375 borgere
703 videomøder

Login:
Jan = 978
Feb = 1134
Mar = 906

Videoopkald:
Jan = 102
Feb = 137
Mar = 46

Kalenderaftaler:
Jan = 1064
Feb = 1191
Nar = 1505

Fokus områder:
Man vælger måske kun mellem 70-80 mennesker og spørger om de vil have VTC eller varme hænder, og så kører man fuld skrue på at implementere det der. Det gav mange flere videomøder tiltrods for at det var færre borgere.

Tidlig opsporing, støtteværktøjer og skærmbesøg benyttes i FK. 

Tværsektorielt samarbejde er guld. IT er godt til at levere fakta.
Opdragende effekt ved at man kan komme direkte i kontakt med fx diabetes 'specialister', når man har en telemedicinsk løsning i og med at der sidder folk i et call center der har ekspertise inden for præcis dette område. 
Man slipper altså for at tage kontakt til sygehuset og bliver sendt videre igen og igen.

De logger tidspunkt og varighed af opkaldet.

\textbf{Anvendelse} 
\\
Skal være chrome browser.\\
Al trafik er krypteret (https)- en masse med password der skal være en vis længde. 
De har også en APP, som så åbner en chromebrowser. 

Video og tale hænger sammen. De er altid synkroniseret og hvis en af dem ikke virker, så stopper den anden også med at virke. \\
Den platform Appinux har kan bruges til mange ting. Tidlig opsporing, skærmbesøg, telemedicin(direkte måling af kol, hjerte mv.), teamtavler, telesundhed.\\
Køberen kan selv bestemme hvilke dele der skal være på platformen. 
Der er mere end 70 moduler i Appinux platformen, som man kan vælge imellem.\\
Det kan styres hvilke moduler den enkelte skal have adgang til, således at alle ikke har adgang til de samme data.

Hardware/software: \\
De ved ikke hvad minimumskravet er.\\
De har nogle anbefalinger - fx ikke under android 4.02 \\
Web RTC understøttet af google hangout(standard) som de bruger. De bruger gerne standarder hos Appinux!\\
Video standard(google hangout) lavede til almindelige mennesker. \\
Favrskov købte tidlig opsporing af Appinux og ikke video!!!!
Men de har fået lov at bruge det.

Minimumskrav til internetforbindelse:
Ikke et reelt krav til app og browser men der skal 500 k/bit/sek til for at der kan køre video.\\
Benytter også virtuel bitrate. \\
Kører på alle bredbåndsforbindelser.
Ligger et simkort i en tablet, giver det til borgeren og så skal det fungere.

Først gemt lokalt på medarbejderens telefon og så sendt når der er hul igennem.\\
Det samme gælder ikke for borgeren da man ikke må gemme på borgerens telefon. 
Appinux har valgt ikke at gøre det, så borgeren skal være online for at det kan gennemføres. Det medfører at Appinux er sikre på at borgeren tager kontakt til sygehyset hvis han/hun er i fare. Det er en grundlæggende sikkerhedsforanstaltning.\\
Hvis patienten ikke svarer sendes der en hjemmeplejer ud til dem.



Ansvarlig for serveren:


Support:
Det kører i cloud.


\textbf{Formål} 
\\


\textbf{Udbredelse} 
\\


\textbf{Erfaring og udfordringer} 
\\


\textbf{Dækningsproblemer} 
\\


\textbf{Demonstration} 
\\


\textbf{Andre løsninger/udviklingsplaner} 
\\
Alle kan i princippet lave apps til appinux' system, men de skal have en rolle og adgang til de forskellige ting. Databehandleraftale.

\textbf{Økonomi} 
\\
SKI aftale.

Appinux leverer denne pakke:
Abonnement
 - Drift. Besværligt med apps. Kommunen tager selv ansvar for systemet/produktet. Hvis kommunen opdaterer fx 40 tablets hvorefter de ikke fungerer, så er det kommunens problem. Appinux har alt virtuelt.  
 - Support: (redmine). Ingen stor support. De kommer ikke fysisk ud. De har heller aldrig oplevet efterspørgsel på det. Problemet modtages af en dansker og sendes videre til en 'medarbejder' i udlandet i tilfælde af at problemet er meget teknisk. 
 - Vedligehold
 - Videreudvikling: 4 releases om året. 
 Kommunerne har 2 systemer. Presystem, uddannelsessystem og Produktionssystem.
 Der kommer en opdatering til presystemet og så tester de det, og kommer med input til hvordan det fungerer, og så laver Appinux eventuelle ændringer.\\
 På den måde tester kommunen deres opsætning i sammenspil med den nye opdatering.
 Appinux tester selv 3 uger før der begynder at komme kunder på. Det er tovejs kommunikation, når der skal laves en opdatering.\\
 Favrskov skal sige til Appinux at de gerne vil opdatere, men de må ikke være mere end to opdateringer bagud.\\
 Uddannelsessystemet bruges til eventuelle nye medarbejdere. 
Kommunerne kan selv vælge hvilken type abonnement de gerne vil have. Fx om de vil have en MDM pakke med, således at Appinux ????står for opdateringen?????.

Man betaler pr. måned for et abonnement:
Betaling foregår pr. antal aktive brugere.
50 brugers løsning med telemedicin(kalender, dataopsamling, video):
10.000 kr. pr. måned

Samarbejdspartnere:
Video og træning fx (exolife, aidq)


2 prismodeller - tjek SKI 2.19 (cloud) og 2.07 med TDC
SE cloudfactory - Appinux ligger under dette firma:
- Grundmodel som er basis. 6500 kr. for 0-50 brugere
- Tidlig opsporing. 2000 kr.
- Telesundheds modulet. 3000 kr.
500 - 750
750 - 1200


Appinux har ansvaret for at serveren kører. Serveren kører i Skyen, men den står ved T26, som har adgang til sundhedsdatanettet. 


Favrskov har fået en pakke til en vis pris.
En sosu medarbejder om måneden ca.!
2 indgangsfees(startpakken):
 - Instalation, 15.000
 - Implementering: opsætning, uddannelse, 57.500

Gratis fagkonsulenter.

Server:


Appinux' kode er frit tilgængeligt, men man skal have et eller andet???





Standarder:
Kommunerne har en særlig standard - fællessprog 1 og 2. De er ved at udvikle 3'eren.
De har lavet deres egen standard.
Den ligger Appinux sig op af. 

Apparater = continua health alliance
Video = web RTC
Sundhedsbaserede integrationer = HL7 - herunder FHIR
Lagring, dansk ift. telemedicinske data = XDS-lagring. IHE profiler(navn, sted, tid, måling mv.) der kan gemmes i XDS. Det er en international standard som DK gerne vil rulle på nationalt plan.
Personjournal = PHMR (personal health monitoring record)



Sikkerhed:
Alle data er krypteret.
Alt på medarbejdersiden er logget, så man kan se hvad de laver. 
Der er nogle superbrugere der kan kontrollere, om folk holder sig inde for deres beføjelser. 
Man kan som borger se hvem der har kigget på dine personlige data, og klage hvis en ukendt læge, sygeplejerske el. har været inde og kigge ens profil.
Pårørende kan få adgang til data, hvis borgeren giver adgang til det. 
Samtykke fra borgeren styrer hvem der har adgang til hvilke data.
Der er mulighed for at brugeren kan trykke på "husk password" på deres tablet, og på den måde tjekke deres personlige data. Der kunne laves noget fingeraflæsning eller lignende. 


Største udfordring:
Implementering fra en fysisk ydelse til en virtuel ydelse.
De ansatte kan evt. miste deres arbejde, og de skal samtidig omvendes fra deres uddanelse om fysisk kontakt til video.	




