

\section{Dato: 18.04.2016}
\hrule

\textbf{Fremmødte:} Lise, Sara, Jakob, Melissa og Mette fra Netplan \\
\textbf{Fraværende:} Jeppe og Mohammed 
\\
\\
\textbf{Dagsorden}
\begin{itemize}
	\item Status 
	\item Organisering i gruppen 
	\item Opfølgning på mødet med Appinux
	\item Har i den nødvendige viden
	\item Behov for teknikmøde vedr. video? 
\end{itemize}

\textbf{Referat} 
\\



Favrskov bruger "Fælles service center" til deres sår-projektet. Disse tester ikke (startede første januar), men det kunne være noget, der bliver aktuelt. De er implementeringspartner, kan man sige.   

\textbf{Status}\\
Vi er okay med og netplan synes også at vi er på sporet. \\ \\
\textbf{TOBS}
TOBS (tidlig opsporing af begyndende sygdomme). Appinux er de første og (måske eneste) i dk, der har lavet et it (app), hvor man benytter TOBS og registrerer disse værdier. \\ \\
\textbf{SKI}
Stattens og kommunernes indkøbsaftaler. Gør det letter for kommunerne at handle med ham - intet udbud er nødvendigt! Tjek om Viewcare er med i det. 


\textbf{Organisation}
Skriv til Karin om de har noget dokumentation om deres arbejdsgange i hjemmeplejen hos hadstens sundhedscenter. Mette vil også spørger Per Lund Jensen på onsdag, da hun har møde med ham. Hun spørge ind til økonomi. Vi har lavet nogle spørgsmål til Mette, som hun kan spørge ham om.  
\\ \\
De bekræfter at de nogle gange har svært ved at overskue, om de forskellige teknologier kan snakke sammen - og derfor vil de gerne have en stradegi om telesundhed.

Kontakt: Anne Cecilie Greve - arbejder både i fælles service center og farvskov kommune. acg@sygdjurs.dk  








