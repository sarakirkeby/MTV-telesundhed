\chapter{Mødereferat}

\section{Dato: 24.02.16}
\hrule

\textbf{Fremmødte:} Lise, Melissa, Jakob, Jeppe, Sara og Mohamed
\textbf{Fraværende:} Ingen
\\
\\
\textbf{Dagsorden}
\item Hvad består denne MTV af? 
\item Hvad er det vidre forløb? 
\item Forventningsafstemning med vejleder
\item Faste møder med vejleder efter påske
\item Forberedelse til møde med NetPlan 
\item Aftale med møde med NetPlan

\textbf{Referat} 
\\
Gå til Bente før Jesper, da han godt kan have andre meninger end Bente på det område, og det kan godt være at der er uoverensstemmelser med meninger.\\
Vi må aldrig bevæge os væk fra fokusområdet. Vi skal altid være ening med NetPlan omkring hvad det er vi skal. Det der mange gange sker med telemedicin er at man glemmerat overveje mange ting (f.eks. kompitabilitet, begrænsninger osv.). \\
Der skal lavet et møde med Stefan Wagner, som arbejder med telemedicin. Ingeniørhøjskolen mener at folk skal have at vide hvad det er de skal bruge. \\
Dem som tager beslutningen skal kunne tage den ud fra det vi skriver i rapporten. Der kan godt komme effekter af midler som man ikke ser komme. Det er vores job at vise eksempler på uforudsete virkninger af produkter. Dette kan vi finde igennem litteratursøgningen. Skotland og England ville være gode at finde eksempler fra, fordi det let kan overføres til de danske forholde. Det er vigtigt at man beskriver sideeffekterne og virkningerne af alternativerne også, f.eks. brocuhrer, fordi argumentation liver bedre. Det handler om at finde løsninger som øger livskvalitet. \\
I en MTV er der sammenhænge imellem de forskellige emner, disse kaldes dialektikker. Disse skal også beskrives dybdegående. Det ville være super ideelt at vi fandt 12 antal rapporter som alle peger på det samme, og derefter påpeger eventuelle forskelle. Virker denne teknologi med nuværende arbejdsmetoder (ex. transportmiddel). Vi går ind og vurderer hvor stor videnskabeligvægt og værdi har denne rapport? \\
De MTV'er vi laver er ikke nødvendigvis evidensbaserede, men den kan skabe evidens.\\
\\Vi skal ud og snakke med Netplan om hvilket fokuspunkt, de gerne vil have vi snakker om. Det er vigtigt når vi møder kommunerne at vi holder styr på mødet. Vi skal lave en problemformulering, en dagsorden og begynde at indsamle en masse information omkring dette emne. Vi kan godt begynde at lave en interessantanalyse.\\
Det er ikke utænkeligt at vores problemformulering bliver ændret hen af vejen.\\
Vi har faste møder tirsdag klokken 12 som udgangspunkt. Husk at vidresende mail til Bente og Jesper.\\

