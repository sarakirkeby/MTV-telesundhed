

\section{Dato: 11.03.2016}
\hrule

\textbf{Fremmødte:} Alle samt Mette fra Netplan og 3 fra Favrskov Kommune
\textbf{Fraværende:} 
\\
\\
\textbf{Dagsorden}
\begin{itemize}
	\item Præsentation af opgave og deltagere
	\item Præsentation af løsning
		\item Forskellige anvendelsesformål: tryghedsbesøg, medicinadministration
		\item Antal borgere, der bruger løsningen og plan for yderligere udbredelse
		\item Drift af løsningen
		\item Dækningsproblemer? 
		\item Erfaringer og udfordringer indtil nu
	\item Demonstration og løsningen, arbejdsgang
\end{itemize}

\textbf{Referat} 
\\
\textbf{Hvem?}
\\
Camilla Lind: 30 timer ugentligt til velfærdsteknologi internt og eksternt i kommunen. 7 timer ugentligt til bevilling af hjælpemidler i kommunen. 

Per Jensen: fuldmægtig på ældreområdet.  

Karin Juhl: sygeplejerske  22 timer ugentligt til telemedicin (anden kollega 10 timer) og 15 timer ugentligt til vagttelefon. Alle henvendelser omkring detaljer vedr. virtuel hjemmepleje skal ske til Karin.

Mette Dalsgaard: senior konsulent og repræsentant for Netplan Care.

Lise, Jeppe, Mohamed, Melissa, Sara, Jacob.

\textbf{Hadsten hjemmepleje efterspørger kommunal telesundhedsstrategi}
\\
Hvad koster det, hvem beslutter, hvad findes der ellers af løsninger. Hvor er vi som kommune på vej hen. Hvad skal medarbejderne arbejde med udgangspunkt i? Ud ad til – over for resten af DK; hvilken strategi arbejder vi ud fra?

\textbf{Telemedicin i Hadsten}
\\
Visitationsmyndighed, der står for bevilling af ydelser til borgere. Det bliver visitationens opgave at tildele videoopkald i stedet for fysiske besøg. 

Der er ikke tale om flere eller nye servicetilbud! Men en omstrukturering af eksisterende ydelser, der kan omlægges fra fysiske besøg til videoopkald. 

Der er tale om konvertering af fysiske besøg til videoopkald.

Pilotprojekt startet op – færdig med implementering i maj i alle fire distrikter. Målet er at visitationen overtager fra oktober og tilbyder borgere videoopkald som erstatning for fysiske besøg

Det er svært at finde ydelser, fordi de ydelser der tilbydes i dag er meget afhængige af, at der skal være en plejeperson fysisk tilstede. Man skal have visiteret en ydelse, før det overhovedet kan vurderes, om ydelsen kan ændres til videoopkald. 

Hadsten ældreområde har en forventning om, at skærmen kommer lettere ud til borgere, der ikke før har fået ydelser  end de borgere, der er vant til at få en ydelse på en bestemt måde.  

Teleudbyder: TDC. 

Drift/support: både egen it-afdeling og appinux. Egen it-afdeling har ikke været nok med fra start.


\textbf{Hvilke ydesler kan konverteres til videoopkald?}
\\
Vejledning/struktur i hverdagen, mellemmåltid, psykisk pleje og støtte. Både hjælpe ydelser og sygeplejeydelser. 

Kan bruges til tidlig opsporing af sygdom for at undgå indlæggelser.'

Appinux kan levere løsninger til genoptræning og psykiatri også  pt er pilotprojektet på ældreområdet, men det er en fordel at kunne rulle det ud til andre områder med samme it-system.

Medicingivning.

\textbf{Øvrigt}
\\
KL’s landkort over velfærdsteknologier.

Business region Århus. 

Hvor afgørende er det, at det ikke er samme plejer, der kommer på fysiske besøg vs. forskellige plejere på skærmen? Hypotese om, at det ikke er så afgørende, at det er forskellige ansigter på skærmen, som det er i de fysiske besøg. 

Hentet inspiration fra Viborg Kommune (View Care).

Anden diskussion: nogle borgere kan selv møde op på et sundhedscenter, hvorfor skal disse så have videoopkald? Borgeren sparer tid/ressourcer. Tilgodeser videoopkald borgeren eller personalet? 

For ti år siden kunne man have fået flere borgere på skærmen, fordi man for ti år siden stadig have ”tryghedsbesøg”  sådanne ”overflødige” besøg findes jo ikke i dag. Det findes kun praktiske besøg. Ydelserne afgør, om borgere får besøg. Der findes ikke tryghedsbesøg i Favrskov kommunne.

\textbf{Appinux løsning}
\\
Kører på chrome browser – det er ikke en applikation.
 
Appinux kan levere løsninger til genoptræning og psykiatri også  pt er pilotprojektet på ældreområdet, men det er en fordel at kunne rulle det ud til andre områder med samme it-system.
 
 Call center.
 
Borgeren kan se sig selv på en lille skærm og plejepersonalet på en større skærm på tablet; ligesom skype/facetime mm.
 
 Fordel at kunne se sig selv i skærmen: hvis fx en sygeplejerske har brug for at se et sår eller et pilleglas, så kan borgeren visuelt se sig selv på skærmen og lettere sikre, at det er den rigtige finger eller pilleglas, som borgeren fremviser.
 
 \textbf{Hvordan bruges virtuel hjemmepleje i pilotforsøget i Hadsten?
}
\\
Plejepersonale ringer op til borgeren via videoopkald på tablet. 

Borgeren skal ikke ringe op til sundhedscenteret – videoopkald skal træde i stedet for planlagte fysiske besøg, så det er plejepersonalet, der står for opkaldene. Dette bunder i, at der ikke skal tilbydes merservice til borgeren. Hvis borgeren skal have muligheden for at ringe op, så skal der være personale til at modtage videoopkald. 

Plejepersonalet trækker ydelserne ud af systemet – fx medicingivning – og finder derfra borgerne, som skal tilbydes videoopkald frem for fysiske besøg. 

\textbf{Udfordringer}
\\
Tablets funktioner – at låse disse fast, så borgeren ikke navigerer forkert rundt.

Sikkerhedskrav til telesundhed (og i særdeleshed videoopkald): personfølsomme oplysninger mm. Hvad er kravene egentlig til dette? 

Internetforbindelsen: problemer? Har indtil nu ikke haft problemer. Kører på mobildata net. Eller WiFi. 

2015 kørte et projekt, hvor borgere i Favrskov Kommune kunne gå ind på kommunens hjemmeside og angive, hvis der var dårlig dækning i et bestemt område  der er fokus på at få optimal dækning i hele kommunen  fordel for virtuel hjemmepleje. 

Problematikken omkring betaler af internetforbindelse  denne er kendt fra nødkald. Som hovedregel er det borgerens eget ansvar og pligt at betale for internetforbindelsen. 

Det er en stor udfordring, at systemet og funktionerne ikke på forhånd er grundigt testet af leverandøren før implementering i hjemmeplejen. 

Kommunens egen opgave at teste nye opdateringer. Opdatering lagt på præsystem (uddannelsessystem) som kommunen selv tester, før opdateringen så udrulles til borgerne. Kommunen detekterer mange fejl, som ifølge kommunen skulle være detekteret af appinux før levering. Kommunen bruger mange ressourcer på disse opdateringer (tid, penge mm) 
Dette er en erfaring, der kan gives videre til andre kommuner: tjek aftalen med leverandøren – sørg for at fordele ansvaret mere passende. 

Det er en udfordring at have appinux som leverandør, idet det er en endnu-ikke helt veletableret virksomhed: niveauet af professionalisme og ansvar ikke helt på niveau med øvrige, større leverandører. 

\textbf{På sigt i Hadsten}
\\
Udbygge systemet til praktiserende læger: for at undgå lægebesøg.

På sigt kunne bruge det til opfølgende hjemmebesøg; hvis en sygeplejerske står hos en borger og er i tvivl om et sår, så kan hun bruge Appinux til at ringe med videoopkald til sundhedscenteret for at få en løsning (skal borgeren tilses af læge, kan sygeplejersken på centeret guide ift. sårbehandling mm).











