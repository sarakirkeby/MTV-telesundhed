

\section{Dato: 29.03.2016}
\hrule

\textbf{Fremmødte:} Alle samt Bente B. og Jesper T. \\
\textbf{Fraværende:} Ingen
\\
\\
\textbf{Dagsorden}
\begin{itemize}
	\item Økonomi afsnittets omstændighed
	\item Formål og fokuserede spørgsmål
	\item Muligvis gennemlæsning af baggrundsafsnittet
	\item Møde med S. Wagner
	\item Hvad er næste skridt? 
	\item Ny mødeordning 
\end{itemize}


\textbf{Referat} 
\\
Vi kan evt. sammenligne ViewCare med Appinux. \\
Hvem er det i kommunen der har truffet beslutningen og hvilke overvejelser har man haft ift. at vælge netop den løsning. Er IT folkene eller de sundhedsfaglige folk i det hele taget blevet inddraget i beslutningen om at vælge Appinux? \\
Spørg kommunen om hvem der er ansvarlig for at tage sådanne beslutninger.\\
Hvad mener de med Open Source?? \\
Det er væsenligt at få en definition af Open Source med. \\
Hvilke delkomponenter er Open Source?\\
Hvad betyder det hvis man tilføjer fx en blodtryksmåling til ydelsen. Kan Appinux leverer sådan en løsning og/eller kan man tilkoble andre leverandøres produkter til løsningen?
Continua health alliance - http://www.continuaalliance.org \\
Standarder omkring teknologi.\\


\textbf{Økonomi afsnittets omstændighed} 
\\
Stefan mener at der er noget lusket over at vi skal se bort fra økonomi delen. Han tænker at det er en meget dyr løsning Favrskov kommune har erhvervet sig. 

Bente: \\ Man kan heller ikke udlade økonomi delen. Vi bliver nødt til at forholde os kritisk til den løsning Favrskov kommune har tilkøbt. \\ 
Vi skal også kigge på andre løsninger end den Appinux leverer.\\
Vi skal både påpege positive og negative ting ved løsningen.\\
Hvad kan vi finde ud af om den løsning der er valgt? \\
Vi skal forstå teknikken i løsningen. \\
Vi skal se kritisk på den økonomiske del af det. \\
Hvis der er noget vi ikke kan få adgang til så må vi skrive det i vores MTV, og så komme med eventuelle bud på hvorfor vi ikke må få den information.\\
Vi kan evt. spørge Favrskov kommune om løsningen.\\
Der kan være noget vi kan være nødt til at kappe, hvis vi slet ikke kan finde noget omkring det.\\
Vi skal også forbi det økonomiske aspekt!

Jesper: \\ Vi bestemmer objektiviteten, så vi kan ringe til leverendøren og spørge om prisen, og grave dybere ned i økonomi delen.\\
Hvis vi begynder at grave i økonomi delen, kan det være at de vil give mere information til os omkring.\\
Hvis man ikke kan gennemskue økonomien i deres løsning, er det svært at stole på den. \\
Vi skal spille på Appinux, men vi skal passe på med implicit at stemple Appinux som en utroværdig virksomhed. \\
Vi skal vise hvad det kan og hvad der er godt ved det, og så evt. sammenligne det med en anden teknologi.

Vi skal beskrive infrastrukturen. 24/7 bemanding, bredbånd, kan det køre på mobil/computer/tv, hvem har ansvaret for serveren, sikkerhed omkring oplysninger. \\

\textbf{Formål og fokuserede spørgsmål} 
\\
Vi skal have fokuseret vores MTV spørgsmål! \\
Vi kan spørge hvilke spørgsmål de 3 interessanter gerne vil have belyst, og så kan vi derefter tage stilling til om vi vil inddrage dem i problemformuleringen.\\
Vi kan evt. dele de fokuserede spørgsmål op i mindre og mere præcise spørgsmål ude i de 4 punkter (tek/org/etik/øko). Vi kan lave et bilag, hvor vi skriver minimumskravene ned. Det kræver x m/bit forbindelse og x RAM i computer for at kunne køre programmet/appen. \\

\textbf{Muligvis gennemlæsning af baggrundsafsnittet} 
\\
Vi er på vej i den rigtige retning. \\

\textbf{Møde med S. Wagner} 
\\
Det bliver efter eksamensperioden. \\
Bente og Jesper tager kontakt til Stefan.\\

\textbf{Hvad er næste skridt?} 
\\
Bente: Arbejde på at få noget mere viden gennem artikler, sådan at vi kan få nogle mere fokuserede spørgsmål.\\
Sætte os bedre ind i teknologien, organisationen, etikken og økonomien.\\
Vi skal også kigge nærmere på Appinux' løsning, så vi også er forberedte på mødet med dem. 
Derefter kan vi kontakte kommunen og forklare at vi har nogle ubesvarede spørgsmål som de evt. vil svare på, eller om vi evt. skal komme ud til dem til et møde. \\

\textbf{Ny mødeordning} 
\\
Møde med Jesper d. 5 klokken 14 om spørgsmål til Appinux.\\
Vejledermøde igen d. 15 klokken 9-10 med begge vejlejder\\

Gruppemøde torsdag d. 31 klokken 10.15 i 231K. \\

D. 7/4 møde om bredbånd ved Netpla// 
D. 12/4 møde med Appinux ved Netplan 
