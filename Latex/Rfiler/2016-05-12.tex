\chapter{Mødereferat}

\section{Dato: 12.05.16}
\hrule

\textbf{Fremmødte:} Lise, Melissa, Jakob, Jeppe, Sara og Mohamed \\
\textbf{Fraværende:} Ingen
\\textbf{Vejledere:} Jesper
\\
\\
\textbf{Dagsorden}
\begin{itemize}
	\item Gennemgang af kommentarer fra første udkast \\ 
\end{itemize}


\textbf{Referat} 
\\
Jesper er i tvivl om, hvad vi kigger på. Det er svært at tyde, at vi kigger på medicinadministration og ”husk at spise”-ydelser. \\
Det skal fremgå mere tydeligt, at det er casen i Favrskov Kommune, som vi kigger på. Husk hele vejen igennem at fokusere på MTV’ens fokuserede spørgsmål. 
Baggrundsafsnit: skal handle om, at Favrskov Kommune har valgt denne løsning før det er udarbejdet en MTV eller andet dokumentation for det. Meningen er ikke at vi skal behandle alting (demografisk udvikling mm), men vi skal finde tre-fire punkter, som vi vil kigge på – og dette skal være ud fra den pågældende cases. Det kan være, at fokus skal findes fra litteratursøgningen; hvad belyses som problemer i litteraturen? \\ Find fokus herfra. 
Skær ind til benet. 
Der mangler evidens - det siger videnskabelig forskning  måske vi skal lave fokuserede spørgsmål ud fra, hvad der reelt mangler evidens for? \\


