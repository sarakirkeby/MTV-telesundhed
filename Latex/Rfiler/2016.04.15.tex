\section{Dato: 15.04.2016}
\hrule

\textbf{Fremmødte:} Lise, Sara, Mohammed, Jeppe, Jakob, Bente og Jesper(facetime)
\textbf{Fraværende:} Melissa 
\\
\\
\textbf{Dagsorden}
\begin{itemize}
	\item Appinux mødet
	\item Møde med Mette
	\item Økonomi
	\item Teknologi
	\item Organisation
	\item Fokuserede spørgsmål
\end{itemize}

\textbf{Referat} 
\\
Bente: Deadline er den 30. maj 2016

\textbf{Appinux mødet} 
\\
Karina Kusk (Syddjurs Kommune) fra ST har købt en del af Appinux' løsning, og vi må gerne ringe til hende.\\

Spurgt kommunen om de er en del af Fælles servicecenter?\\

En tegning af opsætning for at forklare hvad Appinux har ansvaret for og hvad Favrskov Kommune har ansvar for. \\

OS2 kommuner - fælles samarbejde om open-source
IT-minds arbejder bl.a. som leverandører\\


\textbf{Møde med Mette} 
\\
Omhandler primært organisation, men muligvis også lidt økonomi. \\


\textbf{Økonomi} 
\\
Fokuseret spørgsmål: Et overblik over hvilke elementer der skal tænkes ind i en ressourceopgørelse.
Begrund hvorfor vi fx ikke vil kigge på selve økonomien(tal), men blot remse op hvad der gør sig gældende. 

Sparer det arbejdsgange eller ej? \\
Vi kan ikke lave et egentlig øko afsnit. \\
Vi kan passende læse nogle af det artikler på ing.dk og se hvad der bliver sagt der. \\
Hvad er det løsningen erstatter? \\
1. nævn de parametre der gælder og som vi kan kigge på.\\
2. Hvor bruger vi komponenterne og hvad koster de?\\
3. Så kan vi lægge sammen og sammenligne de to løsninger.\\

Kig især på et før/efter scenarie! \\

Vi skal ikke nødvendigvis sætte priser på, men istedet belyse de udgifter der er. \\
Vi skal bruge artikler til at diskutere hvorfor der er så stor forskel på hvad der bliver lagt ud, og hvad der egentlig gælder for økonomien. Det kan gøres ved at kigge på alle omkostningerne ved at implementere telemedicin.\\
Få økonomi afsnittet til at hænge sammen med Org, ved at nævne fx arbejdsgange og derefter finde udgifter eller andre økonomiske parametre. \\
Der bliver ofte sat ?-tegn ved de artikler der omhandler økonomi, da der ofte er mange flere udgifter end man lige regner med til at starte med.\\

Kent Møller Petersen - business case - SDU
Søren Keldberg 


\textbf{Tekonologi} 
\\
Fint at remse op hvad der er af forudsætninger for at en teleløsning fungerer. \\
LAV MODELLER!!! 

\textbf{Organisation} 
\\
Kom også ind på implementering her!\\
Her synes Bente at det passer bedst ind!\\
Her kan man også kigge på hvilke ressourcer der gør sig gældende ved telemedicin kontra varme hænder.\\


\textbf{Fokuserede spørgsmål} 
\\
Bente kunne godt lide ideen med at lave overordnede fokuserede spørgsmål!\\
Økonomi: Et overblik over hvilke elementer der skal tænkes ind i en ressourceopgørelse.\\

Lave en formulering på spørgsmålet der ikke ligger op til et ja/nej svar på spørgsmålet.\\
Begynd at lav indledning til de 4 underafsnit, så man får et bedre overblik over hvor opgaven er på vej hen imod.\\

Vi skal have det med i overvejelsen om man misser nogle ting ved VTC fremfor varme hænder.\\
LÆR AF TELESÅR PROJEKTET! \\
Hvad får man ved telemedicin og hvad mister man?\\
Hvor tit skal man ud og tjekke om alle andre ting er iorden?
