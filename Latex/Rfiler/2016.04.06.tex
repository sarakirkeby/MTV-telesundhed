\section{Dato: 06.04.2016}
\hrule

\end{document}\textbf{Fremmødte:} Melissa og Marianne Thomsen, projektleder på Virtuel hjemme- og sygepleje i Viborg kommune
\textbf{Fraværende:} 
\\
\\
\textbf{Dagsorden}
\begin{itemize}
	\item Præsentation af virtuel hjemmepleje i Viborg Kommune
\end{itemize}
 
\\
\textbf{Hvem?}
\\
Marianne Thomsen, projektleder på Virtuel hjemme- og sygepleje i Viborg kommune.
Tlf.nr. 87 87 60 63

\textbf{Referat}
\\
OBS: Til en anden gang…udvalgsformand: politisk. Må IKKE kontaktes. 
Pilotprojekt over tre år siden. 
Tjek Viborg kommunes hjemmeside for at få et godt indtryk: små videoer.
Afholdt en workshop: hvilke borgere/ydelser kan medarbejderne se nyttige.\\ \\
Hvilke ydelser:\\
Medicinadm. (påmindelse om medicintagning) alle som skærmydelse.
Gns.tid (3000 opkald om måneden) – 2,1 min pr. opkald. \\ \\
Medarbejder: meget intens møde, når det er virtuelt. Der er nærhed i det, de skal koncentrere sig, ingen forstyrrelser. Medarbejder kan se gevinst ved det.
Business case: besparelser ved selve besøget og besparelser i kørsel.
Borgeren: de pårørende gjorde modstand i starten – ikke borgeren! 
Diskretion: ingen hjemmeplejebil uden for huset.
+/- fem min på opkaldsaftalerne.
\\ \\
Antabus over skærm: diskretion!! Og disse har et aktivt liv, så disse kan få videoopkald før/efter arbejde. 
Borgere siger: så snakker man om det, som man skal. 
Invasivt med hjemmebesøg: man skal byde og nyde (give kaffe, se pæn ud, rydde op). 
Mindre indgreb i borgerens hverdag med videoopkald. 
Andre risikofaktorer: de borgere som har brug for fysiske besøg.
Kontaktpersonen visiterer – visitationen siger bare, at det er en mulighed med skærm, men kontaktpersonen visiterer. Dermed er bekymringen om ensomhed og andre risikofaktorer væk.
\\ \\Ydelser:\\
Medicingivning
Sygeplejefaglig opfølgningssamtaler: struktur (er du kommet op, spist osv.)  

80 procent er medicintagning. 
Nogle få terminale borgere: egentlig er det de pårørende til at kalde sygeplejersken op i tvivlsspørgsmål. Tablet: så skærmen er mobil. 
347 borgere med skærm i de tre år. 
Flow 117-130 som hele tiden har en skærm. 
Faglig vidensdeling: faglig vejledning. Plejecenteret kan også bruge det. 
View care: krabbe: afløses af 4G (næsten ikke udfordringer fordi 4G er så udbredt i dag). 
Ipads. 
Integreres med sikkerhedskrav på rådhuset (vpn tunnel og ?). 
Agenda: erfaringer fra virtuel hjemme- og sygepleje (alt). 
\\ \\
Den tekniske del: har haft et meget tæt samarbejde med view care. Teknik og support har fungeret optimalt. 
Besparelser: bruges pengene på vedligehold i stedet? Er der reelt set besparelser? 
Borgerne siger nej tak, hvis ikke teknikken virker.  
”A til A” ansigt til ansigt = den tid, man har med borgeren. 
Derudover er der kørsel, ferie, afspadsering, kurser, møder = en pose penge.
Forebyggelse: mener at der er forebyggelse i at borgeren kan selv, undgår indlæggelser osv., fordi borgeren kan blive fulgt tæt.
Terminalpatienter: kan være hjemme og stadig blive fulgt meget tæt. 
