

\section{Dato: 18.05.2016}
\hrule

\textbf{Fremmødte:} Sara og Jakob 
\\
\textbf{Dagsorden}
\begin{itemize}
	\item Telefonmøde med Karin
\end{itemize}

\textbf{Referat} \\
\\
\textbf{Case 1 - manuelt besøg:}\\
\textbf{Forberedelse}\\
Der aftales mellem borger og sundhedsprofessionelle hvornår mødet skal foregå. 
Borgeren bliver skrevet på kørelisten i forhold til aftale. 
Medicinen er opdelt i ugebokse på forhånd. 
Den sundhedsprofessionelle kører ud til borger. 

\textbf{Under}\\
Observerer at borger indtager sin medicin. 

\textbf{Efter}\\
Den sundhedsprofessionelle forlader borger.

Opgaven bliver vinget af som gennemført

\textbf{Hændelser}\\
2-3 gange i døgnet

\textbf{Tid}\\
10 minutter 

\textbf{Medarbejdere indblandet}\\
Hjemmeplejer og Sygeplejerske\\


\textbf{Case 2 - Virtuelt besøg:}\\
\textbf{Forberedelse}\\
Der aftales mellem borger og sundhedsprofessionelle hvornår mødet skal være. 
Borgeren bliver skrevet på den separate køreliste til videoopkald. 
Medicinen er opdelt i ugebokse på forhånd. 
Den sundhedsprofessionelle ringer op. 

\textbf{Under}\\
Observerer at borger indtager sin medicin. 

\textbf{Efter}\\
Den sundhedsprofessionelle afslutter opkaldet. 

Opgaven bliver vinget af som gennemført 

\textbf{Hændelser}\\
2-3 gange i døgnet

\textbf{Tid}\\
2-3 minutter 

\textbf{Medarbejdere indblandet}\\
Hjemmeplejer og Sygeplejerske


Hvordan er arbejdsgangene ændret ved implementering af virtuel hjemmepleje?\\\\
2 valgmuligheder hvis man skulle ud til en borger. Fysisk kørsel eller aftale i klinik. Åbningstil 11 – 12. Kunne reguleres\\
Morgen ydelse kombineres med medicin givning. Prøver ellers at erstatte med videoopkald.  Hjælper med at tænde tablet og åbne appinux. Åbent i 26 timer, men tablet går i dvale efter 30 min. Mister forbindelse og derved skal borgeren gøre noget aktivt for at fortsætte. Der er lavet billedguide. \\
En forhindring er at deres medicin er låst nede, sådan de ikke tager deres forkerte. De leder lige nu efter en elektronisk lås som kan låses op udefra. Det er nogen bestemte der gør det fra hjemmeplejen. \\
De har svært ved at finde borgere som er egnet fordi selve visitationen af ydelser er bestemt af hvor dårlig borgeren er. Det er en vurdering af hvad de kan og der køres rehabiliteringsprogrammer frem for ydelser. Hvis man finder ud af at de ikke kan trænes til det, så gør man det ikke. Borgerne har ikek styr på det selv. Sygeplejeydelserne er sygehus styrede.\\\\ 
Medicinadministration er at hælde det op – Sygeplejersker \\
Medicin givning – Sygeplejere\\\\
På kørselslisten står der tider – Luften imellem de forskellige borgere svarer til kørsel. Nogen ligger meget tæt på hinanden. Man starter sin køreliste når man går ind, og slukker når man går ud. På kørelisten er det bare estimeret. Det tager lige nu lidt længere tid at holde video opkaldene da de stadig har lidt tekniske problemer i form af at borgeren f.eks. ikke kan svare på appen. \\
Det er dem som laver kørelisterne som kører videoopkald.\\\\ 
Hvor mange borgere kalkulerer i til at skulle benytte skærmopkald? Hvor mange er involveret? \\
Cirka 50 borgere. Hvis de har 10 på nu, så er det det, men de har haft 20-30 stykker igennem som har været på. De er ikke på så længe. Enten fordi de ikke har brug for det, eller også bliver de dårligere og går tilbage til akutbesøg. Det er svært at blive visiteret til det. \\\\
Hvor mange tablets har i købt og hvad kostede de? Bruger i dem til andet end videoopkald?\\
De koster 2100 og så et cover til 200. De har købt 25 tablets. Hver gruppe har en, så der er ca 20 borger tablets. Forventet forbrug var 50. Samsung tablet. TabA. \\
De er i gang med at ligge appinux på den som medarbejderne har i forvejen. Disse er Tab4. De kan ringe op til andre medarbejdere hvis de står hos en borger og er i tvivl om noget. \\
Kører 4 mennesker på eget elektronik. \\\\
Hvor lang tid bruger IT-afdelingen på opdatering af systemet? \\
Karin tester den nye version sammen med en anden. Det tager 2 dage at teste. Giver respons og grønt lys og så ligger de det på. Deres egen IT afdeling bliver kun brugt i sammenhæng med Favrskov skrivebord.\\\\ 
Hvor meget transporttid spares der? Og hvad bruges den sparede tid på?\\
Borgerne vil have 2 besøg om dagen, i 365 dage. Sparer måske 3 minutter per gang. \\\\
Hvad vurderes som kørsel? (køreliste) \\
Det er luften imellem de forskellige borgere som vurderes til at være kørsel. Nogen steder er det virtuelt ingen ting. \\\\
Hvad med support? Fælles servicecenter?\\
Den almene support kører på Appinux, men alt indirekte support (kabler programmer osv), har Favrskov kommune selv. De arbejder lige nu på at de kan logge på et skrive bord og have adgang til alt – også appinux. De har samarbejde med IT afdeling, men det er mere Ad Hoc. Ingen fælles servicecenter. \\\\
Ser de nogen besparelser ved videoopkald? Hvad vinder i og hvad taber i? \\
Der spares lidt tid. Plejerne skal ikke have kørepenge for komme ud til borgerne. \\
Det er kommunen der betaler for alt lige nu. Er på kant af lovgivning da en tablet reelt ikke er et hjælpmiddel. \\\\
Hvilken afdeling af Favrskov hører Hadsten sundhedscenter til i? (Nord/Vest)\\
Indre – Dem der er på plejecenter og korttidsafsnit. 
Ydre – Borgere i eget hjem. (Øst: Hadsten og hinnerup 0g vest: hammel thorsøe og ulstrup). 
Der er kun video i ydre områder, men der er tobs i begge dele. \\
Er der sket noget for at forbedre de tekniske problemer i er stødt ind i? (Udfald/Forsinkelser (Spørgeskema)).\\ 
Lige nu kører det på sim kort. De forventede de ville få mange problemer med netværk. Men det har de egentligt ikke oplevet. De havde egentligt købt en krabbe, men de har ikke haft brug for det. Der er eventuelt kommet flere master op. Hvis borgeren selv har haft et elektronisk apparat så har de kunne ligge det på den. \\
Hvis der ikke er mobilnet, så har de brugt borgerens eget internet f.eks. Det er meget fleksibelt program som kan køre på mange forskellige flader. \\\\
Opstartsomkostninger?\\
Aftalen til video opkald er speciel for Favrskov. Spørg Per hvad det gælder med pilotprojektets økonomi

