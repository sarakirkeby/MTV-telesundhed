

\section{Dato: 02.03.16 - Møde med Netplan}
\hrule

\textbf{Fremmødte:} Sara, Melissa, Lise, Mohamed, Jeppe og Jakob\\

\textbf{Fraværende:} 
\\
\\
\textbf{Dagsorden}
\begin{itemize}
	\item Introduktion
	\item MTV's fokusområde
	\item Interessenter
\end{itemize}

\textbf{Referat} 
\\
\textbf{Facts} 
\\
 - MTV'en skal laves i samarbejde med \textbf{Favrskov Kommune.}\\
 - \textbf{Appinux} = leverandør\\
 - Skal vi vide mere om bredbånd, så tag fat i Netplan!\\

Netplan er eksperter inden for:
Kommunikation, infrastruktur og bredbånd. \\
Netplan Care lavet i 2013 - herunder telesundhed.\\
Der findes registreret omkring 400 initiativer til telemedicin. Så kører de, men så bliver de lukket ned igen fordi de er svære at få til at køre ordenligt - Til dels pga. teknologien, til dels pga. organisation. \\\\
Kommunen får et større og større pres på sig fordi folk har færre sengedage(supersygehuse med færre pladser). Kommunen får altså patienterne "tilbage". \\
Telemedicin: færre indlæggelser, hurtigere rehabilitering.
\\\\
Mange skjulte gevinster - naboen ser ikke at der holder en hjemmepleje bil ude foran huset.\\\\
Hvad er virtuel hjemmepleje? \\
Hjælp til medicin indtag, kontrol(husket at drikke), tryghedskald.\\

Fredag d. 11.03.2016 (formiddag) og 29.03.2016(formiddag) og 01.04.2016(formiddag) og torsdage efter påsken.



England og skotland er længst fremme med telemedicinske løsninger. 


Video-delen er det sprængende punkt. Mest følsomme ift. brugeroplevelsen. Det er tungt og kompleks. Netplan vil holde et møde og forklare uddybbende om problematikken. \\
Video codec der kan følge med ustabile internetforbindelser. Hvis ikke det er gældende - ligesom ved 'Online Velfærd' så stoler brugerne ikke på produktet. \\

Tilstrækkelig båndbredde:\\
Digital velfærd - brug af båndbredde.\\
Rapport til erhvervsstyrelsen, som Netplan har lavet.\\\\

\textbf{MTV's fokusområde}\\
\textbf{Fokus:} Virtuel hjemmepleje - udvikle en strategi for telesundhed i Favrskov Kommune.
\\\\
Referenceprojekt. Det første de laver for en kommune inden for telemedicin.\\
En strategi for telesundhed(). En strategi for hvilken vej de skal gå.
Vi skal vurdere Fagerskov kommunes løsning!! \\
Hvad er det for en hverdag kommunen har? Hvor meget kan vi, og hvad kan lade sig gøre? Man ikke bare tage en rapport fra én kommune og kopiere den til en anden, pga. forskellige politiske ambitioner i kommunerne. \\
Vi kan lave problemformuleringen om/rette den til. \\
På den måde kan vi finde ud af om vi er enige med Netplan og har forstået deres problemformulering korrekt. 
Netplan laver en paraplystrategi og som en del af den skal vi gå i dybden med telesundhedsprojektet og finde ud af hvad man kan opnå med det. Vores MTV er bare en del af den store strategi. \\ 
Vi skal kigge på deres virtuelle hjemmepleje og hvordan den virker.\\
MTV'en skal give en retning til kommunen. \\
Vi arbejder med en løsning der ikke er færdig implementeret. Vi skal kigge på hvordan man bedst muligt kan implementere den fuldstændigt. Vi laver en MTV på et udviklingsforløb. 
 
Planlagt aktivitet: workshop med fagerskov kommune, netplan og os.
Vores projekt: Hvad er der i praksis i fagerskov kommune og hvad er der andre steder som kan evt. kan bruges i fagerskov kommune? Hvilke løsninger findes der?
Vi skal supplere deres evalueringsrapport. Den har Netplan fået!
Fagerskov kommune har næsten ingenting på skrift! 
Telemedicin er meget på forsøgsbasis, fordi kommunerne har så meget andet at se til med udskiftning af it systemer.\\
\\\\
\textbf{Videoløsninger}
\\
Sammenligne forskellige videoløsninger fra andre kommuner der også har arbejdet med VH.
Teknologi og implementering.\\
3/4 aspekter i MTV.\\
Økonomien kommer ikke højsædet.\\
De 3 andre vægter højere!

\textbf{Interessanter}
Kommunen, borgere, apinux, netplan, os. 
\\